\section{First Order Unification}

 Unification is widely used in automated reasoning, logic programming
 and programming language type system implementation.


\subsection{First-order terms}

Given a set of \textit{variable symbols} $X = \{x,y,z,\cdots \}$ , a
set of distinct \textit{constant symbols} $ C = \{ a,b,c,\cdots \}$ a
set of distinct \textit{function symbols} $F = \{f,g,h, \cdots
\}$. Term is defined as any expression that can be generated by a
finite number of applications of the following rules:


\begin{enumerate}
\item Basis: any variable , and also any constant  is a term
\item Induction: if $t_1, \cdots, t_k$ are terms then $f(t_1,\cdots,
  t_k)$ is term for finite $k, 0 < k$.
\end{enumerate}


For brevity, the constant symbols are regarded as function symbols
taking zero arguments, the induction rule is relaxed to allow terms
with zero arguments , and $a()$ is regarded as syntactically equal to
$a$ . Mathematicians fix the arity of a function symbol while
typically in syntactic unification problems, a function symbol may
have any number of arguments, and possibly may have different number
of arguments in different arguments.

First order unification is the syntactic unification of first-order
terms, while higher order unification is the unification of
higher-order terms. First-order unification is especially widely used
in logic programming, programming language type system design,
especially in type inferencing algorithms based on the Hindley-Milner
type system, and automated reasoning. Higher-order unification is also
widely used in proof assistants, for example Isabelle and Twelf, and
restricted forms of higher-order unification (higher-order pattern
unification) are used in some programming language implementations,
such as lambdaProlog, as higher-order patterns are expressive, yet
their associated unification procedure retains theoretical properties
closer to first-order unification. Semantic unification is widely used
in SMT solvers and term rewriting algorithms.

It is well known that if two terms have a unfier, they also have a
\textit{most general unifier}
\subsection{Substitution}

A substitution is defined as a finite set of mappings from variables
to terms where each mapping must be \textit{unique}, because mapping
the same variable to two different terms would be ambiguous. A
substitution may be applied to a term u and is written $u\{ x_0
\rightarrow t_0, \cdots, x_k \rightarrow t_k \}$, which means
\textit{simultaneously} replace every occurrence of each variable
$x_i$ in the term $u$ with the term $t_i$ for $0 \le i \le k $. E.g.
$f(x,a,g(z),y)\{x \rightarrow h(a,y), z \rightarrow b \} = f(h(a,y), a
, g(b),y)$. A unifier U is called a \textit{most general unifer} for
$L$, if $\forall$ $U'$ of $L$, $\exists$ substitution $s$, $subst(U',L) =
subst(s, subst(U,L))$ .

\subsection{Unification in Various areas}

Unification in Prolog
\begin{itemize}
\item A variable which is uninstantiated-i.e. no previous unifications
  were performed on it-can be unified with an atom, a term, or another
  uninstantiated variable, thus effectively becoming its alias. In
  many modern Prolog dialects and in first-order logic, a variable
  cannot be unified with a term that contains it; this is the so
  called occurs check.

\item Two atoms can only be unified if they are identical.


\item Similarly, a term can be unified with another term if the top
  function symbols and arities of the terms are identical and if the
  parameters can be unified simultaneously. Note that this is a
  recursive behavior.

\end{itemize}
Unification in HM type inference 

\begin{itemize}
\item Any type variable unifies with any type expression, and is
  instantiated to that expression. A specific theory might restrict
  this rule with an occurs check.
\item Two type constants unify only if they are the same type
\item Two type constructions unify only if they are applications of
  the same type constructor and all of their component types
  recursively unify.
\end{itemize}

Due to its declarative nature, the order in a sequence of unifications
is (usually) unimportant.


\subsection{Occurs check}

If there is ever an attempt to unify a variable $x$ with a term with a
function symbol containing $x$ as a strict subterm $x=f(\cdots,
x,\cdots)$, $x$ would have to be an infinite term, which contradicts
the strict definition of terms that requires a \textit{finite} number
of applications of the induction rule. e.g $x=f(x)$ does not represent
a strictly valid term.

\subsection{Unification Examples}

\begin{tabular}{|c|c|c|}
\hline
Prolog Notation & Unify Substitution & Explaination \\
\hline 

f(X)=f(Y) & X $\rightarrow$ Y & X and Y are aliased \\
f(g(X),X) = f(Y,a) & X $\rightarrow$ a, Y $\rightarrow$ g(a) & \\
X = f(X) & should fail & enforced by \textit{occurs check} \\
X=Y,Y=a & X $\rightarrow$ a , Y $\rightarrow$ a & \\
\hline 
\end{tabular}

\subsection{Algorithm}
Unification algorithms can either perform occurs checks as soon as a
variable has to be unified with a non-variable term or postpone all
occurs checks to the end. The first kind check is \textit{inline} and
the second is called \textit{post} occurs checks. The second performs well.
\begin{eqnarray}
G \cup  \{t  \sim t\}  \Rightarrow G \\
G \cup \{ f(s_0, \cdots ,s_k )  \sim f(t_0, \cdots, t_k)\}  \Rightarrow
G \cup \{s_0 \sim t_0, \cdots, s_k \sim t_k \}\\
G \cup \{f(s_0, \cdots, s_k) \sim g(t_0,\cdots, t_m ) \} \Rightarrow
\perp if f \neq g \vee  k \neq m\\
G \cup \{ x \sim t \} \Rightarrow
G{x\rightarrow t} \cup {x \sim t} if x \in Vars(G) \wedge x \notin Vars(t) \\
G \cup \{ x \sim f(s_0, \cdots, s_k) \} \Rightarrow \perp if x \in
Vars(f(s_0, \cdots, s_k ) )
\end{eqnarray}
The set of variables in a term $t$ is written as $Vars(t)$, and the et
of variables in all terms on \verb|LHS| or \verb|RHS| of potential
equations in a problem \verb|G| is written as \verb|Vars(G)|. For
brevity, constant symbols are regarded as function symbols having zero
arguments.

Robinson Algorithm
\begin{verbatim}
Function robOccursCheck(x,t,delta)
INPUT:
  Variable x, term t, substitution delta
OUTPUT:
  false or true
BEGIN
  let S be a stack, initially containing t
  while (S is non-empty) do
    t := pop(S);
    foreach variable y in t do
      if x = y then
        return false
      if y is bound in delta  then
        push y <* delta onto S
    od 
  od
  return true
END

FUNCTION ROB(s,t)
INPUT:
  Term s and t
OUTPUT:
  substitution or failure
BEGIN
  let S be an empty stack of pairs of terms, initially containing
(s,t)
  let delta be the empty substitution
  while (S is non-empty) do
     (s,t) := pop (S);
     while (s is a variable bound by delta) s := s <* delta
     while (t is a variable bound by detal) t := t <* delta
     if s <> t then
       case(s,t) of
         (x,y) => add x -> y to delta
         (x,u) =>
               if robOccursCheck(x,u,delta)
               then
                 add x -> u to delta
                 apply x -> u to each term in delta 
               else failure
         (u,x) =>
               if robOccursCheck(x,u,delta)
               then
                 add x -> u to delta
                 apply x -> u to each term in data
               else failure

         (f(s1,..,sn),f(t1,...,tn)) =>
               push (s1,t1), ... , (sn,tn) onto S
      end     
  od
  return delta
END
\end{verbatim}
  