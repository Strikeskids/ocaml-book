\subsection{Developing Applications with Objective Caml}

\begin{enumerate}
  \item caveat
    \begin{enumerate}
    \item + (modulo the boundary, \emph{will not be checked})
    \item $1.0/0.0 \rightarrow \infty $
    \item $+. -. *. /. **$  mod ceil floor sqrt exp log log10 cos sin tan acos asin atan  
    \item $asin 3.14  \rightarrow nan    $
    \item \verb|char_of_int 255| $\rightarrow$ \verb|'\255'| (can not display)
    \item \verb|char_of_int int_of_char string_of_int int_of_string|
      \verb|string_of_int 2551 -> ``2551''|
    \item string (length $\le 2^{24} - 6$ )
    \item == (\textit{physical equal}) (=, != <>)


\begin{alternate}
true == true;;
- : bool = true
# 3 == 3;;
- : bool = true
# 1. == 1.;;
- : bool = false
\end{alternate}

    \item int * int * int is different from (int * int ) * int
    \item unreasonable parametric equality \verb|(=) : 'a -> 'a -> bool|
    \item recursive declaration

\begin{redcode}
let rec ones = 1 :: ones;;
\end{redcode}

\begin{bluecode}
val ones : int list =
  [1; 1; 1; 1; 1; 1; 1; 1; 1; 1; 1; 1; 1; 1; 1; 1; 1; 1; 1; 1; 1; 1; 1; 1; 1;
   1; 1; 1; 1; 1; 1; 1; 1; 1; 1; 1; 1; 1; 1; 1; 1; 1; 1; 1; 1; 1; 1; 1; 1; 1;
   1; 1; 1; 1; 1; 1; 1; 1; 1; 1; 1; 1; 1; 1; 1; 1; 1; 1; 1; 1; 1; 1; 1; 1; 1;
   1; 1; 1; 1; 1; 1; 1; 1; 1; 1; 1; 1; 1; 1; 1; 1; 1; 1; 1; 1; 1; 1; 1; 1; 1;
   1; 1; 1; 1; 1; 1; 1; 1; 1; 1; 1; 1; 1; 1; 1; 1; 1; 1; 1; 1; 1; 1; 1; 1; 1;
   1; 1; 1; 1; 1; 1; 1; 1; 1; 1; 1; 1; 1; 1; 1; 1; 1; 1; 1; 1; 1; 1; 1; 1; 1;
   1; 1; 1; 1; 1; 1; 1; 1; 1; 1; 1; 1; 1; 1; 1; 1; 1; 1; 1; 1; 1; 1; 1; 1; 1;
   1; 1; 1; 1; 1; 1; 1; 1; 1; 1; 1; 1; 1; 1; 1; 1; 1; 1; 1; 1; 1; 1; 1; 1; 1;
   1; 1; 1; 1; 1; 1; 1; 1; 1; 1; 1; 1; 1; 1; 1; 1; 1; 1; 1; 1; 1; 1; 1; 1; 1;
   1; 1; 1; 1; 1; 1; 1; 1; 1; 1; 1; 1; 1; 1; 1; 1; 1; 1; 1; 1; 1; 1; 1; 1; 1;
   1; 1; 1; 1; 1; 1; 1; 1; 1; 1; 1; 1; 1; 1; 1; 1; 1; 1; 1; 1; 1; 1; 1; 1; 1;
   1; 1; 1; 1; 1; 1; 1; 1; 1; 1; 1; 1; 1; 1; 1; 1; 1; 1; 1; 1; 1; 1; 1; 1;
   ...]
\end{bluecode}


\begin{redcode}
 let special_size l = 
    let rec size_aux prev = function 
      |[] -> 0 
      |_ :: l1  -> if List.memq l1 prev then 1 else 1 + size_aux (l1::prev) l1 in size_aux [l]  l;;
    \end{redcode}
\begin{bluecode}    
  val special_size : 'a list -> int = <fun>
\end{bluecode}

\begin{alternate}
# special_size ones;;
- : int = 1
# let rec twos = 1 :: 2 :: twos in special_size twos;;
- : int = 2
# special_size [];;
- : int = 0
\end{alternate}  
\item combine patterns \\
  p1 | .. |  pn (all name is forbidden within these patterns) 
 'a' .. 'e' 

 \begin{alternate}
let test 'a' .. 'e' = true;;
^^^^^^^^^^^^^^^^^
 \end{alternate}

\begin{bluecode}
Warning 8: this pattern-matching is not exhaustive.
Here is an example of a value that is not matched:
'f'
val test : char -> bool = <fun>
\end{bluecode}

    \item records

\begin{alternate}
type complex = {re:float;img:float};;
type complex = { re : float; img : float; }
# let add {re; img} {re; img} = 3;;
val add : complex -> complex -> int = <fun>
# let add {re; img} {re; img} = {re = re +. re; img = img +. img};;
val add : complex -> complex -> complex = <fun>
 \end{alternate}

    \item \emph{redefinition marsks the previous one, while values of the masked types
        still exist, but it now turns to be an abstract type }
    \item exception
      \begin{enumerate}
      \item \verb|Match_failure Division_by_zero Failure|
      \item exception Name of t -- monomorphic , extensible sum Type \\
        when pattern match your exception, its type should be fixed 
      \item control flow
      \end{enumerate}
    \item {\bf disagree over interface} \\
      when toplevel loads the same module (only the name is the same),
      it will check the interface is equal, this sucks since ocaml has
      flat namespace for module 
    \end{enumerate}
  \item sharing \\
    for structured values, it will be sharing , however,
    \emph{vectors of floats don't share}

\begin{alternate}
let a = Array.create 3 0.;;
val a : float array = [|0.; 0.; 0.|]
# a.(0)==a.(1);;
- : bool = false
\end{alternate}


  \item weak type variables \\

\begin{alternate}
  let b = ref []
  (* b should '_a list ref, since b is not pure, cannot be shared *)
  let a = []
  (* a : 'a list *) 
  let a = None
  (* a : 'a option *)n
  let a = Array.create 3 None
  (* '_a option array *)
# type ('a,'b) t ={ch1 : 'a list; mutable ch2 : 'b list};;
type ('a, 'b) t = { ch1 : 'a list; mutable ch2 : 'b list; }
# let v = {ch1=[];ch2=[]};;
val v : ('a, '_b) t = {ch1 = []; ch2 = []}     
\end{alternate}
 \textit{mutable sharing conflicts with polymorphism}

  \item library
    \begin{enumerate}
    \item List \\

\begin{bluecode}
      @ length hd tl nth rev append rev_append concat flatten
      iter map rev_map left_fold fold_right iter2 map2 rev_map2
      fold_left2 fold_right2 for_all exists for_all2 exists2 
      mem memq find filter partition assoc assq remove_assoc remove_assq
      split combine sort statble_sort fast_sort merge
    \end{bluecode}

\begin{alternate}    
# List.assq 3 [3,4;1,2];;
- : int = 4
# List.assq 3. [3.,4;1.,2];;
Exception: Not_found.
\end{alternate}

    \item Array \\
      \verb|Array.create_matrix creates Non-Rectangular matrices|

\begin{bluecode}
length get set make create init -- when you don't want to initialize
make_matrix (int->int->'a -> 'a array array) create_matrix;
append concat sub copy fill ('a array -> int -> int -> 'a -> int)
blit (Array.Labels.blit), to_list, of_list map iteri mapi fold_left
fold_right sort stable_sort fast_sort unsafe_get unsafe_set copy
\end{bluecode}

    \item IO \\

\begin{bluecode}
open_in open_out close_in close_out input_line
input : Batteries.Legacy.in_channel -> string -> int -> int -> int = <fun> 
output: Batteries.Legacy.out_channel -> string -> int -> int -> unit =<fun> 
read_line print_string print_newline print_endline
\end{bluecode}

    \item stack (imperative data structure actually)

\begin{bluecode}
exceptin Empty
create
type 'a t = { mutable c : 'a list }
(* mutable to delay initialization *)
push pop top clear copy is_empty length iter enum copy
of_enum print
module Exceptionless
  top : 'a t -> 'a option, pop
\end{bluecode}

    \item stream \textbf{imperative}

\begin{bluecode}
'a t
exception Failure
exception Error of string
from
of_list of_string of_channel iter empty peek junk count npeek
iapp icons ising lapp lcons lsing
sempty slazy dump npeek
\end{bluecode}

      syntax extension (for my experience, use it in shell, but not in
      tuareg toplevel)
\begin{redcode}
  let concat_stream a b = [<a;b>]
\end{redcode}
\begin{bluecode}
val concat_stream :
  'a Batteries.Stream.t -> 'a Batteries.Stream.t -> 'a Batteries.Stream.t =
\end{bluecode}
   expression not preceded by an \` considered to be sub-stream
   destructive pattern matching (camlp5 or extended parser can merge)
   consumed (error), failure
    \item Array List String Hashtbl Buffer Queue
    \item Sort

\begin{redcode}
module X = Sort ;;
\end{redcode}

\begin{bluecode}
module X :
  sig
    val list : ('a -> 'a -> bool) -> 'a list -> 'a list
    val array : ('a -> 'a -> bool) -> 'a array -> unit
    val merge : ('a -> 'a -> bool) -> 'a list -> 'a list -> 'a list
  end
\end{bluecode}

    \item Weak (vector of weak pointers) abstract type

\begin{bluecode}
sig
  type 'a t = 'a Weak.t
end 
\end{bluecode}


    \item Printf

\begin{bluecode}
%t -> (output->unit)
%t%s -> (output->unit)->string->unit
\end{bluecode}
they all should be processed at \textbf{compile time}


    \item Digest \\
      hash functions return a fingerprint of their entry (reversible) 

\begin{bluecode}
   val string : string -> t -- fingerprint of a string
   val file : string -> t -- fingerprint of a file 
\end{bluecode}

    \item Marshal estimate data size

\begin{alternate}
type external_flag = No_sharing | Closures

let size x = x |> flip Marshal.to_string [] |> flip Marshal.data_size 0;;           ;;
val size : 'a -> int = <fun>
# size 3;;
- : int = 1
# size 3.;;
- : int = 9
# size "ghsogho";;
- : int = 8
# size "ghsogho1";;
- : int = 9
# size "ghsogho1ah";;
- : int = 11
# size 111;;
- : int = 2
\end{alternate}


    \item Sys

\begin{bluecode}
os_type interactive word_size max_string_length
max_array_length time argv getenv command file_exists
remove rename chdir getcwd 
\end{bluecode}

\begin{alternate}
# float (Sys.max_string_length ) /. (2. ** 57.);;
- : float = 0.999999999999999889
\end{alternate}


    \item Arg Filename Printexc
    \item Printexc

\begin{redcode}
# module P = Printexc;;
\end{redcode}

\begin{bluecode}
module P :
  sig
    val to_string : exn -> string
    val catch : ('a -> 'b) -> 'a -> 'b
    val get_backtrace : unit -> string
    val record_backtrace : bool -> unit
    val backtrace_status : unit -> bool
    val register_printer : (exn -> string option) -> unit
    val pass : ('a -> 'b) -> 'a -> 'b
    val print : 'a BatInnerIO.output -> exn -> unit
    val print_backtrace : 'a BatInnerIO.output -> unit
  end
\end{bluecode}


    \item Num
    \item \verb|Arith_status|

\begin{redcode}
# module X = Arith_status;;
\end{redcode}
\begin{bluecode}
module X :
  sig
    val arith_status : unit -> unit
    val get_error_when_null_denominator : unit -> bool
    val set_error_when_null_denominator : bool -> unit
    val get_normalize_ratio : unit -> bool
    val set_normalize_ratio : bool -> unit
    val get_normalize_ratio_when_printing : unit -> bool
    val set_normalize_ratio_when_printing : bool -> unit
    val get_approx_printing : unit -> bool
    val set_approx_printing : bool -> unit
    val get_floating_precision : unit -> int
    val set_floating_precision : int -> unit
  end
\end{bluecode}


    \item Dynlink \\
      choice at execution time, load a new module and hide the
      code code (hot-patch)
      actually (\#load is kinda hot-patch), however to write it in programs
      \emph{more flexible} than \#load , load requires its name are fixed, and
      load will check .mli file, Dynlink \textbf{does not} do this check, while when you
      want to do X.blabla, it still checks, so still don't work, only side
      effects will work.

\begin{redcode}
#direcotry "+dynlink";;
#load "dynlink.cma";;
Dynlink.loadfile "test.cmo";;
\end{redcode}

    \end{enumerate}

  \item syntaxes
  \item expr

\begin{bluecode}
exp	::=value-path  -- value-name or module-path.value-name
 	| constant  
 	| ( expr )  
 	| begin expr end  
 	| ( expr :  typexpr )  
 	| expr ,  expr  { , expr } -- tuple
 	| constr  expr  -- constructor
 	| `tag-name  expr  -- polymorphic variant
 	| expr ::  expr  -- list 
 	| [ expr  { ; expr } ]  
 	| [| expr  { ; expr } |]  
 	| { field =  expr  { ; field =  expr } }  
 	| { expr with  field =  expr  { ; field =  expr } }  
 	| expr  { argument }+ -- application  
 	| prefix-symbol  expr  -- prefix operator
 	| expr  infix-op  expr  
 	| expr .  field  
 	| expr .  field <-  expr  -- still an expression
 	| expr .(  expr )  
 	| expr .(  expr ) <-  expr  
 	| expr .[  expr ]  
 	| expr .[  expr ] <-  expr  
 	| if expr then  expr  [ else expr ]  
 	| while expr do  expr done  
 	| for ident =  expr  ( to |  downto ) expr do  expr done  
 	| expr ;  expr  
 	| match expr with  pattern-matching  
 	| function pattern-matching  
 	| fun multiple-matching  -- multiple parameters matching
 	| try expr with  pattern-matching  
 	| let [rec] let-binding   { and let-binding } in  expr  
 	| new class-path  
 	| object class-body end  
 	| expr #  method-name  
 	| inst-var-name  
 	| inst-var-name <-  expr  
 	| ( expr :>  typexpr )  
 	| ( expr :  typexpr :>  typexpr )  
 	| {< inst-var-name =  expr  { ; inst-var-name =  expr } >}  
 	| assert expr  
 	| lazy expr  
 
argument::=expr  
 	| ~ label-name  
 	| ~ label-name :  expr  
 	| ? label-name  
 	| ? label-name :  expr  
 
pattern-matching::=
 [|] pattern [when expr]-> expr { |pattern  [when expr] ->  expr }  
 
multiple-matching::= { parameter }+  [when expr]-> expr  
 
let-binding::=pattern =  expr  
 	| value-name  { parameter }  [: typexpr] =  expr  
 
parameter::=pattern  
 	| ~ label-name  
 	| ~ ( label-name  [: typexpr] )  
 	| ~ label-name :  pattern  
 	| ? label-name  
 	| ? ( label-name  [: typexpr]  [= expr] )  
 	| ? label-name :  pattern  
 	| ? label-name : (  pattern  [: typexpr]  [= expr] )
\end{bluecode}        
      
\begin{alternate}
  let f ?test:(Some x ) y = x + y;;
  ^^^^^^^^^^^^^^^^^^^^^^^^^
\end{alternate}

\begin{bluecode}
Warning 8: this pattern-matching is not exhaustive.
Here is an example of a value that is not matched:
None
val f : ?test:int -> int -> int = <fun>
\end{bluecode}

  \item pattern

\begin{bluecode}
pattern	::=	value-name  
 	| _  
 	| constant  
 	| pattern as  value-name  
 	| ( pattern )  
 	| ( pattern :  typexpr )  
 	| pattern |  pattern  
 	| constr  pattern  
 	| `tag-name  pattern  
 	| #typeconstr-name  -- object ?
 	| pattern  { , pattern }  
 	| { field =  pattern  { ; field =  pattern } }  
 	| [ pattern  { ; pattern } ]  
 	| pattern ::  pattern  
 	| [| pattern  { ; pattern } |]  
 	| lazy pattern
\end{bluecode}

  \item toplevel-phrase

\begin{bluecode}
toplevel-input::= { toplevel-phrase } ;;  
 
toplevel-phrase::=definition  
 	| expr  
 	| #ident  directive-argument  
 
directive-argument::=epsilon
 	| string-literal  
 	| integer-literal  
 	| value-path
defition ::= let [rec] let-binding {and let-binding}
        | external value-name : typexpr = external-declartion
        | type-definition
        | exception-defition
        | class-definition
        | classtype-definition
        | module module-name {(module-name : module-type)} [:module-type] = module-expr
        | module type module-name = module-type
        | open module-path
        | include module-expr 
\end{bluecode}

  \item type-definition

\begin{bluecode}

type-definition	::= type typedef  { and typedef }  
 
typedef	::= [type-params]  typeconstr-name  [type-information]  
 
type-information::=
  [type-equation] [type-representation]{ type-constraint }  
type-equation::= = typexpr  
 
type-representation::=
          = constr-decl  { | constr-decl }  
 	| = { field-decl  { ; field-decl } }  

type-params::=	type-param  
 	| ( type-param  { , type-param } )  
 
type-param::=	' ident  
 	| + ' ident  
 	| - ' ident  
 
constr-decl::=	constr-name  
 	| constr-name of  typexpr  { * typexpr }  
 
field-decl::=	field-name :  poly-typexpr  
 	| mutable field-name :  poly-typexpr  
type-constraint	::=constraint ' ident =  typexpr
\end{bluecode}

\begin{alternate}
# type t;;
type t
\end{alternate}

\item  interoperating with C

  Difficutilies 
  \begin{enumerate}
  \item Machine reperesentation of data
  \item GC \\
    calling a c function from ocaml must not modify the memory in ways incompatible
    with ocaml gc.
  \item Exceptions \\
    C does not support exceptions, different mechanisms for aborting computations,
    this complicates ocaml's exception handling
  \item sharing common resources \\
    input-output. each language maintains its own input-output buffers.
  \end{enumerate}

  Communications
  \begin{enumerate}
  \item external declarations \\
    it associates a c function definition with an ocaml name, while giving the
    type of the latter.

    \begin{bluetext}
      external caml_name : type = "C_name"
      val caml_name : type
    \end{bluetext}
    both workds, but in the latter case, calls to the c function \textit{first go}
    through the general function application mechanism of ocaml. This is slightly
    less efficient, but hides the implementation of the function as a c function.
  \item external functions with more than five arguments
    \begin{bluetext}
      external caml_name : type = "C_name_bytecode" "C_name_native"
    \end{bluetext}
  \end{enumerate}

  
\end{enumerate}

\subsubsection{chap7 Development Tools}
\label{sec:chap7-devel-tools}
\begin{enumerate}

\item Command names  \\
  
  \begin{tabular}{|c|c|}
    \hline
    ocaml & toplevel top \\
    \hline
    ocamlrun & bytecode interpreter \\
    \hline
    ocamlc & bytecode batch compiler \\
    \hline
    ocamlopt & native code batch compiler \\
    \hline
    ocamlc.opt & \textit{optimized} bytecode batch compiler \\
    \hline
    ocamlopt.opt & \textit{optimized} native code batch compiler \\
    \hline
    ocamlmktop & new \textit{toplevel} constructor \\
    \hline
  \end{tabular}

  The optimized compilers are themselves compiled with the Objective Caml native compiler.
  They compile \textit{faster} but are otherwise \textit{identical} to their unoptimized counterparts.
\item compilation unit \\
  For the interactive system, the unit of compilation corresponds to a phrase of the language. For the batch compiler, the unit of compilation is two files: the source file, and the interface file
  
  \begin{tabular}{|c|c|}
    \hline
    extension & meaning \\
    .ml & source \\
    .mli & interface \\
    .cmi & compiled interface \\
    .cmo & object file (byte) \\
    .cma & library object file(bytecode) \\
    .cmx & object file (native) \\
    .cmxa & library object file(native) \\
    \hline
    .c & c source \\
    .o & c object file (native) \\
    .a & c library object file (native) \\
    \hline
  \end{tabular}

  
  The \textit{compiled interface} is used for both the bytecode and native code compiler.

\item ocamlc \\
  
  \begin{tabular}{|c|c|}
    \hline
    -a & construct a runtime library \\
    -c & compile \textit{without} linking \\
    -o name\_of\_executable & specify the name of the executable \\
    -linkall & link with \textit{all} libraries used \\
    -i & \textit{display all } compiled global declarations \\
    -pp command & preprocessor \\
    -unsafe & turn off index checking \\
    -v & display version \\
    -w list & choose among the list the level of warning message \\
    -impl file & indicate that \textit{file} is a caml source(.ml) \\
    -intf file & as a caml interface(.mli) \\
    -I dir & add directory in the list of directories \\
    \hline
    -thread & light process \\
    -g, -noassert & linking \\
    -custom, -cclib, -ccopt, -cc & standalone executable \\
    -make-runtime, -use-runtime & runtime \\
    -output-obj & c interface \\
  \end{tabular}

  warning messages.

  \begin{tabular}{|c|c|}
    A/a & enable/disable all messages \\
    F/f & partial application in a sequence \\
    P/p & incomplete pattern matching \\
    U/u & missing cases in pattern matching \\
    X/x & enable/disable all other messages \\
    M/m and V/v & for hidden object \\
  \end{tabular}
  the compiler chooses the (A) by default.
  turn off some warnings sometimes is helpful, for example
  \begin{bluetext}
	ocamlbuild -cflags -w,aPF top_level.cma    
  \end{bluetext}

\item ocamlopt   \\
  \begin{tabular}{|c|c|}
    -compact & optimize the produced code for space \\
    -S & keeps the assembly code in a file \\
    -inline level & set the aggressiveness of inlining \\
  \end{tabular}

\item Toplevel
  \begin{tabular}{|c|c|}
    -I dir & adds the directory \\
    -unsafe & no bounds checking \\
  \end{tabular}
\item ocamlmktop \\
  it's ofen used for pulling native object code libraries(typically written in C) into
  a new toplevel.
  \textit{
    -cclib libname, -ccopt optioin, -custom, -I dir -o exectuable
  }

  \begin{bluetext}
    ocamlmktop -custom -o mytoplevel graphics.cma \
    -cclib -I/usr/X11/lib -cclib -lX11
  \end{bluetext}
  
  This \textit{standalone} exe(-custom) wil be \textit{linked} to the library X11(libX11.a) which in turn will be looked up in the path \textit{/usr/X11/lib}

  A standalone exe is a program that \textit{does not } depend on OCaml installation to run.
  The OCaml native compiler produces standalone executables by default. But without \textit{-custom} option, the bytecode compiler produces an executable which requires the \textit{bytecode interpreter ocamlrun}

  \begin{redcode}
ocamlc test.ml -o a
ocamlc -custom test.ml -o b
\end{redcode}

\begin{bluecode}
-rwxr-xr-x   1 bob  staff    12225 Dec 23 16:31 a
-rwxr-xr-x   1 bob  staff   198804 Dec 23 16:31 b
\end{bluecode}

\begin{alternate}
bash-3.2$ cat a | head -n 1
#!/Users/bob/SourceCode/ML/godi/bin/ocamlrun
\end{alternate}
% $

without \textit{-custom}, it depends on \textit{ocamlrun}. With \textit{-custom}, it contains the \textit{Zinc} interpreter as well as the program bytecode, this file can be executed directly or copied to another machien(using the same CPU/Operating System).

Still, the inclusion of machine code means that stand-alone executalbes are not protable
to other systems or other architectures.

\item optimization \\
  It is necessary to not create \textit{intermediate closures} in the case of application on several arguments. For example, when the function \textit{add} is applied with two integers, it is not useful to create the first closure corresponding to the function of applying add to the first argument. It is necessary to note that the creation of a closure would \textit{allocate} certain memory space for the environment and would require the recovery of that memory space in the future. \textit{Automatic memory recovery} is the second major performance concern, along with environment.


\item chap10 Program Analysis Tool \\
  \begin{enumerate}
  \item ocamldep \\

    
    \begin{tabular}{|c|c|}
      -I & add dir \\
      -impl,-intf & \\
      -ml(i)-synonym <e> & cosider <e> as a synonym of .ml(i) extension \\
      -modules & Print module dependencies in raw form(not suitable for make) \\
      -native & generate dependencies for a pure native-code project \\
      -slash & for windows \& unix \\
    \end{tabular}

    
\begin{redcode}
ocamldep -modules *.ml      
\end{redcode}

\begin{bluecode}
ta.ml: Array Printf
tb.ml: Array Ta
\begin{bluecode}

  \begin{redcode}
ocamldep *.ml    
\end{redcode}


\begin{bluecode}
ta.cmo:
ta.cmx:
tb.cmo: ta.cmo
tb.cmx: ta.cmx
\end{bluecode}

other examples

\begin{bluetext}
ocamlfind ocamldep -modules dir_top_level_util.ml > dir_top_level_util.ml.depends
ocamlfind ocamldep -pp 'camlp4of -parser pa_mikmatch_pcre.cma' -modules dir_top_level.ml > dir_top_level.ml.depends
\end{bluetext}
  \end{enumerate}
\end{enumerate}
%%% Local Variables: 
%%% mode: latex
%%% TeX-master: "../master"
%%% End: 




