If you have \textit{Godi} installed, then you have already compiler
libs installed.

\begin{bashcode}
ls `ocamlfind ocamlc -where`/../compiler-lib  
\end{bashcode}

You can play with compiler lib in the toplevel

\begin{ocamlcode}
  #directory ``+../compiler-lib'' ;; (** for cmi file*)
  #load ``toplevellib.cma'';;
\end{ocamlcode}
\captionof{listing}{Compiler lib in toplevel \label{compiler-toplevel}}

\section{module Printtyp}

This module mainly export some printting functions for ocaml ast.

\begin{ocamlcode}
  val longident: formatter -> Longident.t -> unit
  val ident: formatter -> Ident.t -> unit
  val type_expr: formatter -> type_expr -> unit
  val modtype: formatter -> module_type -> unit
  val signature: formatter -> signature -> unit
  val tree_of_modtype_declaration: Ident.t -> modtype_declaration -> out_sig_item
  val modtype_declaration: Ident.t -> formatter -> modtype_declaration -> unit
  val class_type: formatter -> class_type -> unit
\end{ocamlcode}
\captionof{listing}{Printer in compiler lib \label{compiler-printer}}

You can use this library to process \textit{cmi} files, The structure
of \textit{cmi} file is organized as follows

\begin{ocamlcode}
  let ic = open_in_bin filename in
  let magic_len = String.length (Config.cmi_magic_number) in
  let buffer = String.create magic_len in
  really_input ic buffer 0 magic_len ;
  let (name, (sign:Types.signature)) = input_value ic in
  let (crcs : (string * Digest.t) list) = input_value ic in
  let (flags : flags list) = input_value ic in
  close_in ic ;
\end{ocamlcode}
\captionof{listing}{Structure of cmi file}
