\section{Utils}


\subsection{Clflags}

There are a lot of custmoized flags here, which \verb|toploop| respect.
Suppose you turn on \verb|Clflags.dump_rawlambda.val := True;|.


\begin{ocamlcode}
objfiles;
(field 0 (global Clflags!))
- : ref (list string) = {val=[]}
# ccobjs;
(field 1 (global Clflags!))
- : ref (list string) = {val=[]}
# dllibs;
(field 2 (global Clflags!))
- : ref (list string) = {val=[]}
# compile_only;
(field 3 (global Clflags!))
- : ref bool = {val=False}
\end{ocamlcode}


Some flags are really useful since the toploop respsect that you can
watch the effect immediately

\begin{bashcode}
let objfiles = ref ([] : string list)   (* .cmo and .cma files *)
and ccobjs = ref ([] : string list)     (* .o, .a, .so and -cclib -lxxx *)
and dllibs = ref ([] : string list)     (* .so and -dllib -lxxx *)

let compile_only = ref false            (* -c *)
-c  Compile only (do not link)


and output_name = ref (None : string option) (* -o *)
-o <file>  Set output file name to <file>


and include_dirs = ref ([] : string list)(* -I *)
-I <dir>  Add <dir> to the list of include directories


and no_std_include = ref false          (* -nostdlib *)
-nostdlib  Do not add default directory to the list of include directories


and print_types = ref false             (* -i *)
-i  Print inferred interface

and make_archive = ref false            (* -a *)
-a  Build a library


and debug = ref false                   (* -g *)
-g  Save debugging information

and fast = ref false                    (* -unsafe *)
-unsafe  Do not compile bounds checking on array and string access

and link_everything = ref false         (* -linkall *)
-linkall  Link all modules, even unused ones

and custom_runtime = ref false          (* -custom *)
-custom  Link in custom mode

and output_c_object = ref false         (* -output-obj *)
-output-obj  Output a C object file instead of an executable

and ccopts = ref ([] : string list)     (* -ccopt *)
-ccopt <opt>  Pass option <opt> to the C compiler and linker

and classic = ref false                 (* -nolabels *)
-nolabels  Ignore non-optional labels in types

and nopervasives = ref false            (* -nopervasives *)
-nopervasives  (undocumented)

and preprocessor = ref(None : string option) (* -pp *)
-pp <command>  Pipe sources through preprocessor <command>

and ppx = ref ([] : string list)        (* -ppx *)
-ppx <command>  Pipe abstract syntax trees through preprocessor <command>

let annotations = ref false             (* -annot *)
-annot  Save information in <filename>.annot

let binary_annotations = ref false      (* -annot *)
and use_threads = ref false             (* -thread *)
-thread  Generate code that supports the system threads library

and use_vmthreads = ref false           (* -vmthread *)
and noassert = ref false                (* -noassert *)
-noassert  Do not compile assertion checks

and verbose = ref false                 (* -verbose *)
-verbose  Print calls to external commands

and noprompt = ref false                (* -noprompt *)
noprompt in the toplevel

and nopromptcont = ref false            (* -nopromptcont *)
and init_file = ref (None : string option)   (* -init *)
and use_prims = ref ""                  (* -use-prims ... *)
-use-prims <file>  (undocumented)

and use_runtime = ref ""                (* -use-runtime ... *)
-use-runtime <file>  Generate bytecode for the given runtime system

and principal = ref false               (* -principal *)
-principal  Check principality of type inference

and recursive_types = ref false         (* -rectypes *)
and strict_sequence = ref false         (* -strict-sequence *)
-strict-sequence  Left-hand part of a sequence must have type unit
turning the warning 5 into an error


and applicative_functors = ref true     (* -no-app-funct *)
-no-app-funct  Deactivate applicative functors
Roughly: If you apply a functor twice with the same input modules, the
opaque types in the output remain compatible. For instance:
module S1 = Set.Make(String)
module S2 = Set.Make(String)
Now, S1.t and S2.t are type-compatible, although this type is opaque.
(E.g. you can do S1.empty = S2.empty.)
Compare this with:
module Make(X : sig end) = struct type t = Variant end
module M1 = Make(struct end)
module M2 = Make(struct end)
Now, M1.t and M2.t are incompatible - for nominal types like variants
the functors aren't applicative, and each instance is a thing of its
own:
# M1.Variant = M2.Variant;;
Error: This expression has type M2.t but an expression was expected of
type M1.t


and make_runtime = ref false            (* -make-runtime *)
-make-runtime  Build a runtime system with given C objects and libraries

and gprofile = ref false                (* -p *)
and c_compiler = ref (None: string option) (* -cc *)
-cc <command>  Use <command> as the C compiler and linker

and no_auto_link = ref false            (* -noautolink *)
-noautolink  Do not automatically link C libraries specified in .cma files

and dllpaths = ref ([] : string list)   (* -dllpath *)
-dllpath <dir>  Add <dir> to the run-time search path for shared libraries

and make_package = ref false            (* -pack *)
-pack  Package the given .cmo files into one .cmo

and for_package = ref (None: string option) (* -for-pack *)
-for-pack <ident>  Ignored (for compatibility with ocamlopt)

and error_size = ref 500                (* -error-size *)
let dump_parsetree = ref false          (* -dparsetree *)
and dump_rawlambda = ref false          (* -drawlambda *)
and dump_lambda = ref false             (* -dlambda *)
and dump_clambda = ref false            (* -dclambda *)
and dump_instr = ref false              (* -dinstr *)

let keep_asm_file = ref false           (* -S *)
let optimize_for_speed = ref true       (* -compact *)

and dump_cmm = ref false                (* -dcmm *)
let dump_selection = ref false          (* -dsel *)
let dump_live = ref false               (* -dlive *)
let dump_spill = ref false              (* -dspill *)
let dump_split = ref false              (* -dsplit *)
let dump_scheduling = ref false         (* -dscheduling *)
let dump_interf = ref false             (* -dinterf *)
let dump_prefer = ref false             (* -dprefer *)
let dump_regalloc = ref false           (* -dalloc *)
let dump_reload = ref false             (* -dreload *)
let dump_scheduling = ref false         (* -dscheduling *)
let dump_linear = ref false             (* -dlinear *)
let keep_startup_file = ref false       (* -dstartup *)
let dump_combine = ref false            (* -dcombine *)

let native_code = ref false             (* set to true under ocamlopt *)
let inline_threshold = ref 10

let dont_write_files = ref false        (* set to true under ocamldoc *)

let std_include_flag prefix =
  if !no_std_include then ""
  else (prefix ^ (Filename.quote Config.standard_library))
;;

let std_include_dir () =
  if !no_std_include then [] else [Config.standard_library]
;;

let shared = ref false (* -shared *)
let dlcode = ref true (* not -nodynlink *)

let runtime_variant = ref "";;     (* -runtime-variant *)
\end{bashcode}


\subsection{Config}
It's another module which has a lot of configurations. It's generated
by \verb|config.mlbuild|

\subsection{Ccomp}

A module compiling C files and building c libraries. It's mainly
involved in calling external commands.


\subsection{Consistbl}

It's a module for checking consistency of module CRCs.

\subsection{Misc}

Some utlities.

\subsection{Tbl}
A map data structure


\subsection{Terminfo}

\subsection{Warnings}

A module defining different warnings.
