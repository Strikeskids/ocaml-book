\section{Revised syntax}


\begin{alternate}
  '\''
  '''
  let x = 3
  value  x = 42 ; (str_item) (do't forget ;)
  let x = 3 in x + 8
  let x = 3 in x + 7 (expr)

  -- signature
  val x : int
  value x : int ;

  -- abstract module types
  module type MT
  module type MT = 'a

  -- currying functor 
  type t = Set.Make(M).t
  type t = (Set.Make M).t

  --
  e1;e2;e3
  do{e1;e2;e3}

  --
  while e1 do e2 done
  while e1 do {e2;e3 }
  for i = e1 to e2 do e1;e2 done
  for i = e1 to e2 do {e1;e2;e3}

  --
  () always needed

  x::y
  [x::y]
  x::y::z
  [x::[y::[z::t]]]
  x::y::z::t
  [x;y;z::t]

  match e with
  [p1 -> e1
  |p2 -> e2];


  fun x -> x
  fun [x->x]

  
  value rec  fib = fun [
  0|1  -> 1 
  |n -> fib (n-1) + fib (n-2)
  ];


  fun x y (C z) -> t
  fun x y -> fun [C z -> t]
  -- the curried pattern matching can be done with "fun", but
  -- only irrefutable

  -- legall

  fun []

  match e with []

  try e with []


  -- pattern after "let" and "value" must be irrefutable

  let f (x::y) = ...
  let f = fun [ [x::y] ->  ... ]


  x.f <- y
  x.f := y
  x:=!x + y
  x.val := x.val + y

  --
  int list
  list int


  ('a,bool) foo
  foo 'a bool (*camlp4o -str "type t = ('a,bool) foo" -printer r  -> type t = foo 'a bool*)

  type 'a foo = 'a list list
  type foo 'a = list (list a)

  int * bool
  (int * bool )


  -- abstract type are represented by a unbound type variable
  type 'a foo
  type foo 'a = 'b

  type t = A of i | B
  type t = [A of i | B]


  -- empty is legal
  type foo = []


  type t= C of t1 * t2
  type t = [C of t1 and t2]


  C (x,y)
  C x y


  type t = D of (t1*t2)
  type t = [D of (t1 * t2)]


  D (x,y)
  D (x,y)


  type t = {mutable x : t1 }
  type t = {x : mutable t1}


  if a then b
  if a then b else ()


  a or b & c
  a || b && c


  (+)
  \+


  (mod)
  \mod



  (*  new syntax
     it's possible to group together several declarations
     either in an interface or in an implementation by enclosing
     them between "declare" and "end" *)
     
declare
  type foo = [Foo of int | Bar];
  value f : foo -> int ;
end ;


   [<'1;'2;s;'3>]
   [:`1; `2 ; s; `3 :]

   parser [
     [: `Foo  :] -> e 
     |[: p = f :] -> f ]


   parser []
   match e with parser []


   -- support where syntax
   value e = c
     where c = 3 ;


   -- parser
   value x = parser [
   [: `1; `2  :] -> 1 
   |[: `1; `2 :] -> 2 
   ];

   -- object
   class ['a,'b] point
   class point ['a,'b]
   

   class c = [int] color
   class c = color [int]

   -- signature
   class c : int -> point
   class c : [int] -> point 
   

   method private virtual
   method virtual private

   --
   object val x = 3 end
   object value x = 3; end


   object constraint 'a = int end
   object type 'a = int ; end

   -- label type 
   module type X = sig val x : num:int -> bool  end ;
   module type X = sig value x : ~num:int -> bool ; end;

   --
   ~num:int
   ?num:int


\end{alternate}

%%% Local Variables: 
%%% mode: LaTex
%%% TeX-master: "../master"
%%% End: 
