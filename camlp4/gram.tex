\section{Grammar}

We can Re-Raise the exception so it gets \textit{printed} using \textit{Printexc} .  A literal string (like ``foo'') indicates a \textbf{KEYWORD} token. Using it in a grammar \textbf{registers the keyword} with the lexer. When it is promoted as a key word, it will no longer be used as a \textbf{LIDENT}, so for example, the parser parser, will \textbf{break some valid programs} before, because \textbf{parser} is now a keyword. This is the convention, to make things simple, you can find other ways to overcome the problem, but it's too complicated. you can also say (x= KEYWORD) or pattern match syntax (`LINDENT x) to get the actual token constructor. The parser \textbf{ignores} extra tokens after a success.


\textbf{LEVELS}  They can be labeled following an entry, like \textit{expr
  LEVEL "mul"}. However, explicitly specifying a level when calling an
entry might \textit{defeat} the start/continue mechanism.

\textbf{NEXT LIST0 SEP OPT TRY} NEXT refers to the entry being defined at the following level  regardless of assocaitivity or position.
  LIST0 elem SEP sep .
  Both LIST0 and OPT can match the epsilon, but its \textit{priority} is lower.
  For TRY, non-local backtracking, a Stream.Error will be \textit{converted} to
  a Stream.Failure.
  \begin{ocamlcode}
    expr : [[ TRY f1 -> "f1" | f2 -> "f2" ]]
  \end{ocamlcode}
  
\textbf{Nested rule} (only one level  can be applied)
  \begin{ocamlcode}
    [x = expr ; ["+" | "plus" ]; y = expr -> x + y ]
  \end{ocamlcode}
  
\textbf{EXTEND} is an expression (of type unit) \\
It can be evaluated at toplevel, but also inside a function, when
the syntax extension takes place when the function is called.

\textbf{Translated} sample code   (Listing \ref{Gram Transform output})
\inputminted[fontsize=\scriptsize]{ocaml}{code/camlp4/extend_trans/first.ml}
\captionof{listing}{Gram Transform output \label{Gram Transform output}}
If there are unexpected symbols after a correct expression, the trailing symbols are ignored.

\begin{ocamlcode}
let expr_eoi = Grammar.Entry.mk "expr_eoi" ;;
EXTEND expr_eoi : [[ e = expr ; EOI -> e]]; END ;;
\end{ocamlcode}

The keywords are stored \textit{ in a hashtbl}, so it can be updated
\verb|dynamically|.

\textbf{LEVEL} \\
  \begin{bluetext}
  rule ::= list-of-symbols-seperated-by-semicolons -> action 
  level ::=  optional-label optional-associativity
  [list-of-rules-operated-by-bars] 
  entry-extension ::=
  identifier : optional-position  [ list-of-levels-seperated-by-bars ] 
  optional-position ::= FIRST | LAST | BEFORE label | AFTER label |
  LEVEL label  
  \end{bluetext}
\textbf{Grammar modification} When you extend an entry, by default
   the first level of the extension extends the first level of
   the entry.
   Listing \ref{Grammar modification} lists different usage of \textit{EXTEND, EXTEND LAST,  EXTEND LEVEL, EXTEND BEFORE}.

   \inputminted[fontsize=\scriptsize]{ocaml}{code/camlp4/grmmar_modification/first.ml}
   \captionof{listing}{Grammar modification\label{Grammar modification}}

Grammar example
  You can do some search in the toplevel as follows

\begin{ocamlcode}
se (FILTER _* "val" _* "expr" space+ ":" ) "Syntax" ;;
\end{ocamlcode}

\begin{ocamlcode}  
Gram.Entry.print Format.std_formatter Syntax.expr;;
\end{ocamlcode}

\subsubsection{Example: Expr Parse Tree}
Listing \ref{Expr Parse Tree}   is the expr parse tree.
\inputminted[fontsize=\scriptsize]{ocaml}{code/camlp4/expr_parse_tree/first.ml}
\captionof{listing}{Expr Parse Tree\label{Expr Parse Tree}}


A syntax extension of \verb|let try|

\begin{ocamlcode}
let try a = 3 in true with Not_found -> false || false;;
true  
\end{ocamlcode}

First, it uses start parser to parse \textit{let try a = 3 in true
  with Not\_found -> false}, then it calls the cont parser, and the
next level cont parser, etc, and then it succeeds. This also applies
to ``apply'' level.


When you modify the module \textit{Camlp4.PreCast.Gram}, it will be
reflected in the toplevel on the fly. And be careful, it's probably you never come back and need to restart the toplevel.

When two rules overlapping, the EXTEND statement replaces the
old version by the new one and displays a warning. 

\begin{ocamlcode}
se (FILTER _* "warning") "Syntax"
\end{ocamlcode}
\begin{ocamlcode}
type warning = Loc.t -> string -> unit
val default_warning : warning
val current_warning : warning ref
val print_warning : warning
\end{ocamlcode}
  




%%% Local Variables: 
%%% mode: latex
%%% TeX-master: "../master"
%%% End: 
