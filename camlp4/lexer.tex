\section{Camlp4 Lexer}

\begin{ocamlcode}
type camlp4_token =
 | KEYWORD       of string
 | SYMBOL        of string
 | LIDENT        of string
 | UIDENT        of string
 | ESCAPED_IDENT of string
 | INT           of int * string
 | INT32         of int32 * string
 | INT64         of int64 * string
 | NATIVEINT     of nativeint * string
 | FLOAT         of float * string
 | CHAR          of char * string
 | STRING        of string * string
 | LABEL         of string
 | OPTLABEL      of string
 | QUOTATION     of quotation
 | ANTIQUOT      of string * string
 | COMMENT       of string
 | BLANKS        of string
 | NEWLINE
 | LINE_DIRECTIVE of int * option string
 | EOI
\end{ocamlcode}

This lexer is generic since it only embed lexical conventions, it
knows nothing about keywords for instance.  This includes, basic
things: LIDENT: lower case identifier.  UIDENT: upper case identifier.
Integers, and floats: INT: 42, 0xa0, 0XffFFff, 0b1010101, 0O644,
0o644...  INT32: 42l, 0xa0l...  INT64: 42L, 0xa0L...  NATIVEINT: 42n,
0xa0n...  FLOAT: 42.5, 1.0, 2.4e32

Strings and characters with escaping: STRING:

 """, ``foo'', ``bar\n'',``nl=\n, tab=\t, bs=\\ dq=\", sq=', \010,
\xa0'' 
 CHAR: 'a', 'B', '\n', '\, '\n'...  

But also advanced, ones: Quotations and Antiquotations: 

QUOTATION: 
<< foo >> 
<:quot_name< bar >> 
<@loc_name< bar >> 
<:foo@loc_name< bar >>

ANTIQUOT: 
$foo$ 
$anti_name:foo$ 
$`anti_name:foo$

Symbols and escaped identifiers: 
SYMBOL: *, +, +++*\%, \%#@...  
ESCAPED_IDENT: ( * ), ( ++##> ), ( foo ) 

LINE_DIRECTIVE: 
#line 42, 
#foo ``string''...  

And also the layout :
NEWLINE: a newline 
BLANKS: some non newlines blanks


COMMENT: a comment (comments can be nested, strings,
chars and quotations must be well terminated)



\verb|Lexer| was written using ocamllex ,

\begin{ocamlcode}
(**File Lexer.mll*)
    | ( "#"  | "`"  | "'"  | ","  | "."  | ".." | ":"  | "::"
      | ":=" | ":>" | ";"  | ";;" | "_"
      | left_delimitor | right_delimitor ) as x  { SYMBOL x }
    | '$' { if antiquots c
            then with_curr_loc dollar (shift 1 c)
            else parse (symbolchar_star "$") c }
    | ['~' '?' '!' '=' '<' '>' '|' '&' '@' '^' '+' '-' '*' '/' '%' '\\'] symbolchar *
                                                                as x { SYMBOL x }
\end{ocamlcode}

Here's an example :

\inputminted[fontsize=\scriptsize, ]{ocaml}{camlp4/code/lexer_test.ml}
Concerned rules are:
\begin{ocamlcode}
(**cited from Nicolas Pouillard, I could not find source code in
  camlp4 repository *)
 | Some
        ('!' | '%' | '&' | '$' | '#' | '+' | '/' | ':' | '<' | '=' | '>' |
         '?' | '@' | '\\' | '~' | '^' | '|' | '*' as c) ->
        Stream.junk strm__;
        let s = strm__ in reset_buffer (); store c; ident2 s')

and ident2 (strm__ : _ Stream.t) =
    match Stream.peek strm__ with
      Some
        ('!' | '%' | '&' | '$' | '#' | '+' | '-' | '/' | ':' | '<' |'=' |
         '>' | '?' | '@' | '\\' | '~' | '^' | '|' | '*' as c) ->
        Stream.junk strm__; let s = strm__ in store c; ident2 s'

\end{ocamlcode}

In short '?' and '!' are in the same character class but '.' is
treated by the default case. Since the lexer search the longest token
that matches, ``?!'' are packed together but not with a '.' .
