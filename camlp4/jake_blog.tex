\item jake's blog
  \begin{enumerate}
  \item part1 \\
    easy to experiment, using my previous {\bf oco},
    and type

    \begin{bluecode}
oco       
open Camlp4.PreCast ;;
let _loc = Loc.ghost ;;
(** An idea, how about writing another pretty printer, the printer is awful*)
\end{bluecode}

\item part2 \\ 
  just ast transform, easy to experiment in toplevel
  

\begin{redcode}
let cons = ["A"; "B";"C"];;
let tys = Ast.tyOr_of_list (List.map (fun str -> <:ctyp< $uid:str$ >>) cons);;
\end{redcode}

\begin{bluecode}
val tys : Camlp4.PreCast.Ast.ctyp =
  Camlp4.PreCast.Ast.TyOr (<abstr>,
   Camlp4.PreCast.Ast.TyId (<abstr>, Camlp4.PreCast.Ast.IdUid (<abstr>, "A")),
   Camlp4.PreCast.Ast.TyOr (<abstr>,
    Camlp4.PreCast.Ast.TyId (<abstr>,
     Camlp4.PreCast.Ast.IdUid (<abstr>, "B")),
    Camlp4.PreCast.Ast.TyId (<abstr>,
     Camlp4.PreCast.Ast.IdUid (<abstr>, "C"))))
\end{bluecode}
   
\begin{redcode}
(** here you can better understand what ctyp really means, a type
expression, not a top-level struct, cool 
*)
let verify = <:ctyp< A |B |C>>;;
\end{redcode}

\begin{bluecode}
val verify : Camlp4.PreCast.Ast.ctyp =
  Camlp4.PreCast.Ast.TyOr (<abstr>,
   Camlp4.PreCast.Ast.TyId (<abstr>, Camlp4.PreCast.Ast.IdUid (<abstr>, "A")),
   Camlp4.PreCast.Ast.TyOr (<abstr>,
    Camlp4.PreCast.Ast.TyId (<abstr>,
     Camlp4.PreCast.Ast.IdUid (<abstr>, "B")),
    Camlp4.PreCast.Ast.TyId (<abstr>,
     Camlp4.PreCast.Ast.IdUid (<abstr>, "C"))))
   \end{bluecode}

\begin{alternate}   
verify = tys;;
- : bool = true (** amazing result! *)
\end{alternate}

\begin{redcode}
let type_def = <:str_item< type t = $tys$>>;;
\end{redcode}

\begin{bluecode}
val type_def : Camlp4.PreCast.Ast.str_item =
  Camlp4.PreCast.Ast.StSem (<abstr>,
   Camlp4.PreCast.Ast.StTyp (<abstr>,
    Camlp4.PreCast.Ast.TyDcl (<abstr>, "t", [],
     Camlp4.PreCast.Ast.TySum (<abstr>,
      Camlp4.PreCast.Ast.TyOr (<abstr>,
       Camlp4.PreCast.Ast.TyId (<abstr>,
        Camlp4.PreCast.Ast.IdUid (<abstr>, "A")),
       Camlp4.PreCast.Ast.TyOr (<abstr>,
        Camlp4.PreCast.Ast.TyId (<abstr>,
         Camlp4.PreCast.Ast.IdUid (<abstr>, "B")),
        Camlp4.PreCast.Ast.TyId (<abstr>,
         Camlp4.PreCast.Ast.IdUid (<abstr>, "C"))))),
     [])),
     Camlp4.PreCast.Ast.StNil <abstr>)
\end{bluecode}
   
\begin{alternate}
Printers.OCaml.print_implem type_def ;;
type t = | A | B | C;;
let verify = <:str_item< type t = | A | B | C>>;;
\end{alternate}

\begin{bluecode}
val verify : Camlp4.PreCast.Ast.str_item =
  Camlp4.PreCast.Ast.StSem (<abstr>,
   Camlp4.PreCast.Ast.StTyp (<abstr>,
    Camlp4.PreCast.Ast.TyDcl (<abstr>, "t", [],
     Camlp4.PreCast.Ast.TySum (<abstr>,
      Camlp4.PreCast.Ast.TyOr (<abstr>,
       Camlp4.PreCast.Ast.TyOr (<abstr>,
        Camlp4.PreCast.Ast.TyId (<abstr>,
         Camlp4.PreCast.Ast.IdUid (<abstr>, "A")),
        Camlp4.PreCast.Ast.TyId (<abstr>,
         Camlp4.PreCast.Ast.IdUid (<abstr>, "B"))),
       Camlp4.PreCast.Ast.TyId (<abstr>,
        Camlp4.PreCast.Ast.IdUid (<abstr>, "C")))),
     [])),
   Camlp4.PreCast.Ast.StNil <abstr>)
\end{bluecode}
 
\begin{alternate} 
# verify = type_def;;
- : bool = false
\end{alternate}


\begin{redcode}
let match_case = List.map (fun c -> <:match_case< $uid:c$ -> $`str:c$ >>) cons|> Ast.mcOr_of_list ;;
let to_string = <:expr< function $match_case$ >>;;
\end{redcode}

\begin{bluecode}
val to_string : Camlp4.PreCast.Ast.expr =
  Camlp4.PreCast.Ast.ExFun (<abstr>,
   Camlp4.PreCast.Ast.McOr (<abstr>,
    Camlp4.PreCast.Ast.McArr (<abstr>,
     Camlp4.PreCast.Ast.PaId (<abstr>,
      Camlp4.PreCast.Ast.IdUid (<abstr>, "A")),
     Camlp4.PreCast.Ast.ExNil <abstr>,
     Camlp4.PreCast.Ast.ExStr (<abstr>, "A")),
    Camlp4.PreCast.Ast.McOr (<abstr>,
     Camlp4.PreCast.Ast.McArr (<abstr>,
      Camlp4.PreCast.Ast.PaId (<abstr>,
       Camlp4.PreCast.Ast.IdUid (<abstr>, "B")),
      Camlp4.PreCast.Ast.ExNil <abstr>,
      Camlp4.PreCast.Ast.ExStr (<abstr>, "B")),
     Camlp4.PreCast.Ast.McArr (<abstr>,
      Camlp4.PreCast.Ast.PaId (<abstr>,
       Camlp4.PreCast.Ast.IdUid (<abstr>, "C")),
      Camlp4.PreCast.Ast.ExNil <abstr>,
      Camlp4.PreCast.Ast.ExStr (<abstr>, "C")))))
    \end{bluecode}

\begin{alternate}    
Printers.OCaml.print_implem <:str_item<let f = $to_string$ >>;;
let f = function | A -> "A" | B -> "B" | C -> "C";;
\end{alternate}
\begin{redcode}
let match_case2 = List.map (fun c -> <:match_case< $`str:c$ -> $uid:c$
>>) cons|> Ast.mcOr_of_list ;;
\end{redcode}

\begin{alternate}
Printers.OCaml.print_implem <:str_item<let f = function $match_case2$ >>;;
let f = function | "A" -> A | "B" -> B | "C" -> C;;
Printers.OCaml.print_implem <:str_item<let f = function $match_case2$ | _ -> invalid_arg "haha">>;;
let f = function | "A" -> A | "B" -> B | "C" -> C | _ -> invalid_arg "haha";;
\end{alternate}

  anyother way to verify? The output does not seem to guarantee its
  correctness..
  {\bf when you do antiquotation, in the cases of inserting an AST rather
  than a string, usually you do not need tags, when you inserting a
  string, probably you need it.}
\item part3 : quotations in depth
  \begin{bluetext}
[`QUOTATION x -> Quotation.expand _loc x Quotation.DynAst.expr_tag ]    
\end{bluetext}

The `QUOTATION token contains a record including the body of the
quotation and the tag. The record is passed off to the Quotation
module to be expanded. The expander parses the quotation string
starting at some non-terminal(you specified), then runs the result
through the antiquotation expander 

\begin{bluetext}
  |`ANTIQUOT (``exp'' | ``'' | ``anti'' as n) s ->
  <:expr< $anti:make_anti ~c:"expr" n s $>>
\end{bluetext}

The antiquotation creates a special AST node to hold the body of the
antiquotation, each type in the AST has a constructor (ExAnt, TyAnt,
etc.) c here means context.

\begin{bluetext}
  27 matches for "Ant" in buffer: Camlp4Ast.partial.ml
      5:    | BAnt of string ]
      9:    | ReAnt of string ]
     13:    | DiAnt of string ]
     17:    | MuAnt of string ]
     21:    | PrAnt of string ]
     25:    | ViAnt of string ]
     29:    | OvAnt of string ]
     33:    | RvAnt of string ]
     37:    | OAnt of string ]
     41:    | LAnt of string ]
     47:    | IdAnt of loc and string (* $s$ *) ]
     87:    | TyAnt of loc and string (* $s$ *)
     93:    | PaAnt of loc and string (* $s$ *)
    124:    | ExAnt of loc and string (* $s$ *)
    202:    | MtAnt of loc and string (* $s$ *) ]
    231:    | SgAnt of loc and string (* $s$ *) ]
    244:    | WcAnt of loc and string (* $s$ *) ]
    251:    | BiAnt of loc and string (* $s$ *) ]
    258:    | RbAnt of loc and string (* $s$ *) ]
    267:    | MbAnt of loc and string (* $s$ *) ]
    274:    | McAnt of loc and string (* $s$ *) ]
    290:    | MeAnt of loc and string (* $s$ *) ]
    321:    | StAnt of loc and string (* $s$ *) ]
    337:    | CtAnt of loc and string ]
    352:    | CgAnt of loc and string (* $s$ *) ]
    372:    | CeAnt of loc and string ]
    391:    | CrAnt of loc and string (* $s$ *) ];
\end{bluetext}

\begin{alternate}
<:expr< $int: "4"$ >>;;
- : Camlp4.PreCast.Ast.expr = Camlp4.PreCast.Ast.ExInt (<abstr>, "4")
<:expr< $`int: 4$ >>;; (** the same result *)
- : Camlp4.PreCast.Ast.expr = Camlp4.PreCast.Ast.ExInt (<abstr>, "4")
<:expr< $`flo:4.1323243232$ >>;;
- : Camlp4.PreCast.Ast.expr = Camlp4.PreCast.Ast.ExFlo (<abstr>, "4.1323243232")
# <:expr< $flo:"4.1323243232"$ >>;;
- : Camlp4.PreCast.Ast.expr = Camlp4.PreCast.Ast.ExFlo (<abstr>, "4.1323243232")
(** maybe the same for flo *)
\end{alternate}

antiquotation example 

\begin{bluetext}
    match_case:
      [ [ "["; l = LIST0 match_case0 SEP "|"; "]" -> Ast.mcOr_of_list l
        | p = ipatt; "->"; e = expr -> <:match_case< $p$ -> $e$ >> ] ]
    ;
    match_case0:
      [ [ `ANTIQUOT ("match_case"|"list" as n) s ->
            <:match_case< $anti:mk_anti ~c:"match_case" n s$ >>
        | `ANTIQUOT (""|"anti" as n) s ->
            <:match_case< $anti:mk_anti ~c:"match_case" n s$ >>
        | `ANTIQUOT (""|"anti" as n) s; "->"; e = expr ->
            <:match_case< $anti:mk_anti ~c:"patt" n s$ -> $e$ >>
        | `ANTIQUOT (""|"anti" as n) s; "when"; w = expr; "->"; e = expr ->
            <:match_case< $anti:mk_anti ~c:"patt" n s$ when $w$ -> $e$ >>
        | p = patt_as_patt_opt; w = opt_when_expr; "->"; e = expr -> <:match_case< $p$ when $w$ -> $e$ >>
      ] ]
    \end{bluetext}
    
you can see that \verb|match_case0|, if we use the list antiquotation,
the first case in \verb|match_case0| returns an antiquotation with tag
\verb|listmatch_case|,and we get the following expansion

\begin{bluetext}
  value antiquot_expander = object
    inherit Ast.map as super;
    method patt = fun
      [ <:patt@_loc< $anti:s$ >> | <:patt@_loc< $str:s$ >> as p ->
          let mloc _loc = MetaLoc.meta_loc_patt _loc _loc in
          handle_antiquot_in_string s p TheAntiquotSyntax.parse_patt _loc (fun n p ->
            match n with
            [ "antisig_item" -> <:patt< Ast.SgAnt $mloc _loc$ $p$ >>
            | "antistr_item" -> <:patt< Ast.StAnt $mloc _loc$ $p$ >>
            | "antictyp" -> <:patt< Ast.TyAnt $mloc _loc$ $p$ >>
            | "antipatt" -> <:patt< Ast.PaAnt $mloc _loc$ $p$ >>
            | "antiexpr" -> <:patt< Ast.ExAnt $mloc _loc$ $p$ >>
            | "antimodule_type" -> <:patt< Ast.MtAnt $mloc _loc$ $p$ >>
            | "antimodule_expr" -> <:patt< Ast.MeAnt $mloc _loc$ $p$ >>
            | "anticlass_type" -> <:patt< Ast.CtAnt $mloc _loc$ $p$ >>
            | "anticlass_expr" -> <:patt< Ast.CeAnt $mloc _loc$ $p$ >>
            | "anticlass_sig_item" -> <:patt< Ast.CgAnt $mloc _loc$ $p$ >>
            | "anticlass_str_item" -> <:patt< Ast.CrAnt $mloc _loc$ $p$ >>
            | "antiwith_constr" -> <:patt< Ast.WcAnt $mloc _loc$ $p$ >>
            | "antibinding" -> <:patt< Ast.BiAnt $mloc _loc$ $p$ >>
            | "antirec_binding" -> <:patt< Ast.RbAnt $mloc _loc$ $p$ >>
            | "antimatch_case" -> <:patt< Ast.McAnt $mloc _loc$ $p$ >>
            | "antimodule_binding" -> <:patt< Ast.MbAnt $mloc _loc$ $p$ >>
            | "antiident" -> <:patt< Ast.IdAnt $mloc _loc$ $p$ >>
            | _ -> p ])
            | p -> super#patt p ];
    method expr = fun
      [ <:expr@_loc< $anti:s$ >> | <:expr@_loc< $str:s$ >> as e ->
          let mloc _loc = MetaLoc.meta_loc_expr _loc _loc in
          handle_antiquot_in_string s e TheAntiquotSyntax.parse_expr _loc (fun n e ->
            match n with
            [ "`int" -> <:expr< string_of_int $e$ >>
            | "`int32" -> <:expr< Int32.to_string $e$ >>
            | "`int64" -> <:expr< Int64.to_string $e$ >>
            | "`nativeint" -> <:expr< Nativeint.to_string $e$ >>
            | "`flo" -> <:expr< Camlp4_import.Oprint.float_repres $e$ >>
            | "`str" -> <:expr< Ast.safe_string_escaped $e$ >>
            | "`chr" -> <:expr< Char.escaped $e$ >>
            | "`bool" -> <:expr< Ast.IdUid $mloc _loc$ (if $e$ then "True" else "False") >>
            | "liststr_item" -> <:expr< Ast.stSem_of_list $e$ >>
            | "listsig_item" -> <:expr< Ast.sgSem_of_list $e$ >>
            | "listclass_sig_item" -> <:expr< Ast.cgSem_of_list $e$ >>
            | "listclass_str_item" -> <:expr< Ast.crSem_of_list $e$ >>
            | "listmodule_expr" -> <:expr< Ast.meApp_of_list $e$ >>
            | "listmodule_type" -> <:expr< Ast.mtApp_of_list $e$ >>
            | "listmodule_binding" -> <:expr< Ast.mbAnd_of_list $e$ >>
            | "listbinding" -> <:expr< Ast.biAnd_of_list $e$ >>
            | "listbinding;" -> <:expr< Ast.biSem_of_list $e$ >>
            | "listrec_binding" -> <:expr< Ast.rbSem_of_list $e$ >>
            | "listclass_type" -> <:expr< Ast.ctAnd_of_list $e$ >>
            | "listclass_expr" -> <:expr< Ast.ceAnd_of_list $e$ >>
            | "listident" -> <:expr< Ast.idAcc_of_list $e$ >>
            | "listctypand" -> <:expr< Ast.tyAnd_of_list $e$ >>
            | "listctyp;" -> <:expr< Ast.tySem_of_list $e$ >>
            | "listctyp*" -> <:expr< Ast.tySta_of_list $e$ >>
            | "listctyp|" -> <:expr< Ast.tyOr_of_list $e$ >>
            | "listctyp," -> <:expr< Ast.tyCom_of_list $e$ >>
            | "listctyp&" -> <:expr< Ast.tyAmp_of_list $e$ >>
            | "listwith_constr" -> <:expr< Ast.wcAnd_of_list $e$ >>
            | "listmatch_case" -> <:expr< Ast.mcOr_of_list $e$ >>
            | "listpatt," -> <:expr< Ast.paCom_of_list $e$ >>
            | "listpatt;" -> <:expr< Ast.paSem_of_list $e$ >>
            | "listexpr," -> <:expr< Ast.exCom_of_list $e$ >>
            | "listexpr;" -> <:expr< Ast.exSem_of_list $e$ >>
            | "antisig_item" -> <:expr< Ast.SgAnt $mloc _loc$ $e$ >>
            | "antistr_item" -> <:expr< Ast.StAnt $mloc _loc$ $e$ >>
            | "antictyp" -> <:expr< Ast.TyAnt $mloc _loc$ $e$ >>
            | "antipatt" -> <:expr< Ast.PaAnt $mloc _loc$ $e$ >>
            | "antiexpr" -> <:expr< Ast.ExAnt $mloc _loc$ $e$ >>
            | "antimodule_type" -> <:expr< Ast.MtAnt $mloc _loc$ $e$ >>
            | "antimodule_expr" -> <:expr< Ast.MeAnt $mloc _loc$ $e$ >>
            | "anticlass_type" -> <:expr< Ast.CtAnt $mloc _loc$ $e$ >>
            | "anticlass_expr" -> <:expr< Ast.CeAnt $mloc _loc$ $e$ >>
            | "anticlass_sig_item" -> <:expr< Ast.CgAnt $mloc _loc$ $e$ >>
            | "anticlass_str_item" -> <:expr< Ast.CrAnt $mloc _loc$ $e$ >>
            | "antiwith_constr" -> <:expr< Ast.WcAnt $mloc _loc$ $e$ >>
            | "antibinding" -> <:expr< Ast.BiAnt $mloc _loc$ $e$ >>
            | "antirec_binding" -> <:expr< Ast.RbAnt $mloc _loc$ $e$ >>
            | "antimatch_case" -> <:expr< Ast.McAnt $mloc _loc$ $e$ >>
            | "antimodule_binding" -> <:expr< Ast.MbAnt $mloc _loc$ $e$ >>
            | "antiident" -> <:expr< Ast.IdAnt $mloc _loc$ $e$ >>
            | _ -> e ])
      | e -> super#expr e ];
\end{bluetext}

here we see the ambiguity of original syntax,

\begin{bluetext}
<< type t = [ $list:List.map (fun c -> <:ctyp< $uid:c$ >>)$]  >>
\end{bluetext}

in original syntax, it does not know it's variant context, or just
type synonm. (you can add a constructor to make it clear)

\item part4 parsing ocaml itself using camlp4

  \begin{redcode}
Camlp4.Register.loaded_modules;;
\end{redcode}

\begin{bluecode}
- : string list ref =
{Pervasives.contents =
  ["Camlp4ListComprehension"; "Camlp4MacroParser"; "Camlp4MacroParser";
   "Camlp4GrammarParser"; "Camlp4OCamlParserParser";
   "Camlp4OCamlRevisedParserParser"; "Camlp4OCamlParser";
   "Camlp4QuotationExpander"; "Camlp4OCamlRevisedParser"]}
\end{bluecode}

we have to use revised syntax here, because when using quasiquotation,
it has ambiguity to get the needed part, revised syntax was designed
to reduce the ambiguity here . 

The following code is a greate file parsing ocaml itself.
Do not use MakeSyntax below, since it will introduce unnecessary
abstraction type, which makes sharing code very difficult

\begin{redcode}
open Batteries_uni ; 
open Camlp4.PreCast ; 
module MySyntax = Camlp4.OCamlInitSyntax.Make Ast Gram Quotation ;
module M = Camlp4OCamlRevisedParser.Make MySyntax ; (* load r parser *)
(** in toplevel, I did not find a way to introduce such module
    because it will change the state 
*)
module N = Camlp4OCamlParser.Make MySyntax ; (* load o parser*)
value my_parser = MySyntax.parse_implem;
value str_items_of_file file_name = 
  file_name
  |> open_in  
  |> Stream.of_input
  |> my_parser (Loc.mk file_name)
  |> flip Ast.list_of_str_item [] ;

(** it has ambiguity in original syntax, so pattern match 
    will be more natural in revised syntax 
*)
value rec do_str_item str_item tags = 
  match str_item with 
      [ <:str_item< value $rec:_$ $binding$ >> -> 
        let bindings = Ast.list_of_binding binding []
        in List.fold_right do_binding bindings tags 
      |_ -> tags ]
and do_binding bi tags = match bi with 
  [ <:binding@loc< $lid:lid$ = $_$ >> -> 
    let line = Loc.start_line loc in 
    let off = Loc.start_off loc in 
    let pre = "let " ^ lid in 
    [(pre,lid,line,off) :: tags ]
  | _ -> tags ];  


value do_fn file_name = 
    file_name 
    |> str_items_of_file 
    |> List.map (flip do_str_item [])
    |> List.concat ; 
(**use MSyntax.parse_implem*)
value _ = 
  do_fn "/Users/bob/SourceCode/OCaml/Parsing/camlp4/otags_test.ml"
  |> List.iter (fun (a, b, c, d) -> Printf.printf "%s-%s %d-%d \n" a b c d)  ; 
value do_fn_2 fn_2 = fn_2 ;

(**use my syntax *)
(* do_fn "/Users/bob/SourceCode/OCaml/Parsing/camlp4/otags.ml";  *)
(* Exception: Loc.Exc_located <abstr> (Stream.Error "entry [implem] is *)
(* empty"). *)
\end{redcode}

\begin{bluecode}
(* - : list (string * string * int * int) = *)
(* [("let str_items_of_file", "str_items_of_file", 4, 9); *)
(*  ("let do_str_item", "do_str_item", 15, 286); *)
(*  ("let do_binding", "do_binding", 21, 519)] *)
\end{bluecode}


\begin{bluecode}
(** tags *)
"otags.ml" : pp(camlp4rf )
<otags.{cmo,byte,native}> : pkg_dynlink , use_camlp4_full, pkg_batteries
(** be careful, when you use the parser to lift itself, you have to
provide a lot of parsers...
module M4 = Camlp4QuotationExpander.Make MySyntax ;
can make your parser parse itself, great!!
*)
\end{bluecode}

\begin{redcode}
se (FILTER _* "of_") "Stream" ;;
\end{redcode}
\begin{bluecode}
    val of_list : 'a list -> 'a t
    val of_string : string -> char t
    val of_channel : in_channel -> char t
    val of_enum : 'a BatEnum.t -> 'a Stream.t
    val of_input : BatIO.input -> char Stream.t
    val of_fun : (unit -> 'a) -> 'a Stream.t
\end{bluecode}

\item part5 structure item filters \\
  because I use revised syntax, and take a reference of the
  documenation, my ast filter is much nicer than jaked's.
  the documentation of quasiquotation from the wiki page is quite
  helpful 
\begin{bluecode}
value (|>) x f = f x ;
module Make (AstFilters : Camlp4.Sig.AstFilters) = struct 
  open AstFilters ; 
  value code_of_con_names name cons _loc =
    let match_cases = 
      cons |> 
      List.map 
        (fun str -> <:match_case< $uid:str$ -> $str:str$ >>)
      |> Ast.mcOr_of_list  in 
    let reverse_cases = 
      cons |> 
      List.map (fun con -> <:match_case< $str:con$ -> $uid:con$ >>)
        |> Ast.mcOr_of_list in 
    <:str_item< 
      value $lid:(name^"_to_string") $ =
        fun [ $match_cases$ ] ; 
      value $lid:(name^"_of_string") $ = 
        fun [ $reverse_cases$ | x -> invalid_arg  x ] >> ; 
  value rec filter str_item = match str_item with 
      [ <:str_item@_loc< type $lid:tid$  = [ $t$ ] >> -> begin 
        (* [ ] is necessary for revised syntax, 
           otherwise, it will be weird, [] tells it in a list context 
         *)
        try 
         (** good, this can be got from Abstract_Syntax_Tree *)
          let ctys = Ast.list_of_ctyp t [] in 
          let con_names  = 
            List.map (fun [ <:ctyp< $uid:c $ >> -> c 
                          | x -> "FUCK" ]) ctys in 
          let code  = code_of_con_names tid con_names _loc in 
          <:str_item< $str_item$ ; $code$ ; >>
        with 
            [Exit ->  begin 
              print_endline "check " ; 
              str_item end ] 
      end 
      |_ -> begin print_endline "not simple type " ; str_item end ];
  AstFilters.register_str_item_filter filter ;   
end ;
module Id = struct 
  value name = "filter_toy"; 
  value version = "0.1" ; 
end ; 
value _ = 
  let module M =  Camlp4.Register.AstFilter Id Make in 
  () ; 
\end{bluecode}

\begin{bluetext}
"filter.ml" : pp(camlp4rf )
<filter.{cmo,byte,native}> : pkg_dynlink, use_camlp4_full, pkg_batteries
"filter_test.ml" : pp(camlp4of -parser filter.cmo)
\end{bluetext}

the register mechanism should be remembered
\textit{let module M =  Camlp4.Register.AstFilter Id Make in}

we can test our filter as follows \\
\verb|camlp4of -parser _build/filter.cmo filter_test.ml -filter lift -printer o |\\
by the \textbf{lift filter} you can see its \textbf{internal representation}, textual
code does not gurantee its correctness, but the AST representation
could gurantee its correctness.
Built in filters as follows : 
\begin{enumerate}[(a)]
\item fold map
  \begin{bluetext}
  class x = Camlp4MapGenerator.generated ;
  class x = Camlp4FoldGenerator.generated ;
  \end{bluetext}

\item meta \\ 
  lifting function from a type definition -- these functions are what
  \emph{Camlp4AstLifter uses} to lift the AST, and also how
  \emph{quotations are implemented  }
\item LocationStripper (replace location with Loc.ghost) \\
  might be useful when you compare two asts? YES!
  idea? how to use lifter at toplevel, how to beautify our code,
  without the horribling output? (I mean, the qualified name is horrible)
\item Camlp4Profiler \\
  inserts profiling code
\item Camlp4TrashRemover \\
\item Camlp4ExceptionTracer 
\end{enumerate}

\item part6 extensible parser (moved to extensible parser part)

\item part7 revised syntax \\
  revised syntax provides more context in the form of extra brackets
  etc. so that antiquotation works more smoothly.
\item part8, 9 quotation
  \begin{enumerate}[(a)]
  \item Quotation.add  quotation\_expander 

\begin{redcode}
se (FILTER _* "expand_fun") "Quotation";;
\end{redcode}

\begin{bluecode}
type 'a expand_fun = Ast.loc -> string option -> string -> 'a
val add : string -> 'a DynAst.tag -> 'a expand_fun -> unit
val find : string -> 'a DynAst.tag -> 'a expand_fun      
\end{bluecode}

other useful functions 
  \begin{bluecode}
type 'a expand_fun = Ast.loc -> string option -> string -> 'a
val add : string -> 'a DynAst.tag -> 'a expand_fun -> unit
val find : string -> 'a DynAst.tag -> 'a expand_fun
val default : string ref  (* default quotations *)
val parse_quotation_result :
      (Ast.loc -> string -> 'a) ->
      Ast.loc -> Camlp4.Sig.quotation -> string -> string -> 'a
val translate : (string -> string) ref
val expand : Ast.loc -> Camlp4.Sig.quotation -> 'a DynAst.tag -> 'a
val dump_file : string option ref
\end{bluecode}

in previous camlp4, Quotation provides a string to string
transformation, then it default uses Syntax.expr or Syntax.patt to
parse the returned string. following drawbacks
\begin{itemize}
\item needs a \textbf{more} parsing phase
\item the resulting string may be syntactically incorrect, difficult
  to \textbf{debug}
\end{itemize}

\item quotation expander \\ 
  when without antiquotaions, a parser is enought, other things are
  quite mechanical

\begin{bluecode}
open Camlp4.PreCast 
module Jq_ast = struct 
  type float' = float 
  type t = 
      Jq_null 
    |Jq_bool of bool 
    |Jq_number of float' 
    |Jq_string of string 
    |Jq_array of t list 
    |Jq_object of (string*t) list 
end
include Jq_ast 
module MetaExpr = struct 
  (** the generator scans all the types defined in the current module
      then generate code for the last-appearing recursive bundle
  *)
  let meta_float' _loc f = <:expr< $`flo:f$ >>
  include Camlp4Filters.MetaGeneratorExpr(Jq_ast)
  (* due to this can not run in toplevel *)
end 
module MetaPatt = struct 
  let meta_float' _loc f = <:patt< $`flo:f$ >>
  include Camlp4Filters.MetaGeneratorPatt(Jq_ast)  
end 
module MGram = MakeGram(Lexer)
let json_parser = MGram.Entry.mk "json" 
  EXTEND MGram 
  GLOBAL : json_parser ; 
  json_parser : 
    [["null" -> Jq_null 
     |"true" -> Jq_bool true
     |"false" -> Jq_bool false 
     | n = [x = INT -> x | y = FLOAT -> y  ] -> Jq_number (float_of_string n )
     | s = STRING -> Jq_string s 
     | "["; xs = LIST0 SELF SEP "," ; "]" -> Jq_array xs 
     | "{"; kvs = LIST0 [s = STRING; ":"; v = json_parser -> (s,v)] SEP ","; 
       "}" -> Jq_object kvs 
     ]] ; END 
let json_eoi = MGram.Entry.mk "json_eoi"  
  EXTEND MGram 
  GLOBAL: json_eoi ; 
  json_eoi : [[x = json_parser ; EOI -> x ]] ; END 
let test = 
  MGram.parse_string json_eoi (Loc.mk "<string>") 
    "[true,false]"
\end{bluecode}
  
Mechanical installation to get a quotation expander 
\begin{redcode}
module Q = Syntax.Quotation 
(* #directory "/Users/bob/SourceCode/OCaml/Parsing/camlp4/_build";; *)
(* camlp4of -filter meta json.ml -printer o *)
let (|>) x f = f x 
let parse_quot_string _loc s = 
  MGram.parse_string  json_eoi _loc s 
let expand_expr _loc _ s = 
  s 
  |> parse_quot_string _loc 
  |> MetaExpr.meta_t _loc 

(* to make it able to appear in the toplevel *)
let expand_str_item _loc _ s = 
  (**insert an expression as str_item *)
   <:str_item@_loc< $exp: expand_expr _loc None s $ >>
let expand_patt _loc _ s  = 
  s 
  |> parse_quot_string _loc 
  |> MetaPatt.meta_t _loc 

let _  = 
  Q.add "json" Q.DynAst.expr_tag expand_expr ;
  Q.add "json" Q.DynAst.patt_tag expand_patt ;
  Q.add "json" Q.DynAst.str_item_tag expand_str_item ;
  Q.default := "json"

(** make quotation from a parser *)
let install_quotation my_parser (me,mp) name =
  let module Q = Syntax.Quotation in 
  let expand_expr _loc _ s = s |>  my_parser _loc |> me _loc in
  let expand_str_item _loc _ s =  <:str_item@_loc< $exp: expand_expr
  _loc None s $>> in
  let expand_patt _loc _ s = s |> my_parser _loc |> mp _loc in
  Q.add name Q.DynAst.expr_tag expand_expr ;
  Q.add "json" Q.DynAst.patt_tag expand_patt ;
  Q.add "json" Q.DynAst.str_item_tag expand_str_item 

\end{redcode}
\begin{bluetext}
val install_quotation :
  (Camlp4.PreCast.Ast.loc -> string -> 'a) ->
  (Camlp4.PreCast.Ast.loc -> 'a -> Camlp4.PreCast.Ast.expr) *
  (Camlp4.PreCast.Ast.loc -> 'a -> Camlp4.PreCast.Ast.patt) -> string -> unit =
  <fun>  
\end{bluetext}
\begin{bluecode}
"json.ml" : pp(camlp4of -filter meta)
<json.{cmo,byte,native}> : pkg_dynlink, use_camlp4_full
\end{bluecode}
so in the toplevel

\begin{redcode}
#directory "/Users/bob/SourceCode/OCaml/Parsing/camlp4/_build";;
#load "json.cmo" ;
open Json;  (* for Jq_ast module, you can find other ways to work
around this *)
\end{redcode}

\begin{alternate}
 << [ 3 ,4 ]>>;;
- : Json.Jq_ast.t = Json.Jq_ast.Jq_array [Json.Jq_ast.Jq_number 3.; Json.Jq_ast.Jq_number
4.]
\end{alternate}


\item antiquotation expander \\

  the meta filter treat any other constructor \textbf{ending in Ant}
specially 

instead of 
\begin{bluecode}
  |Jq_Ant(loc,s) -> <:expr< Jq_Ant ($meta_loc loc$, $meta_string s$) >>
\end{bluecode}
they have
\begin{redcode}
  |Jq_Ant(loc,s) -> ExAnt(loc,s) 
\end{redcode}

Instead of lifting the constructor, they translate it directly to
ExAnt or PaAnt.

\textbf{Attention, there is no semi or comma required in GLOBAL list,
  GLOBAL: json\_eoi  json ; (just whitespace ) }


  \begin{bluecode}
open Camlp4.PreCast  
module Jq_ast = struct 
  type float' = float 
  type t = 
      Jq_null 
    |Jq_bool of bool 
    |Jq_number of float' 
    |Jq_string of string 

    |Jq_array of t  
    |Jq_object of t 
    |Jq_colon of t * t  (* to make an object *)
    |Jq_comma of t * t  (* to make an array *)
    |Jq_Ant of Loc.t * string 
    |Jq_nil (* similiar to StNil *)
  let rec t_of_list lst = match lst with 
    |[] -> Jq_nil 
    | b::bs -> Jq_comma (b, t_of_list bs)
end 

include Jq_ast 

module MGram = MakeGram(Lexer) 

let json  = MGram.Entry.mk "json"
let json_eoi = MGram.Entry.mk "json_eoi" 


EXTEND  MGram 
  GLOBAL: json_eoi  json ; 
  json_eoi : [[ x = json ; EOI -> x ]]; 

  json : 
    [[ "null" -> Jq_null 
     |"true" -> Jq_bool true 
     |"false" -> Jq_bool false 

     | `ANTIQUOT (""|"bool"|"int"|"floo"|"str"|"list"|"alist" as n , s) -> 
       Jq_Ant(_loc, n ^ ": " ^ s )

     | n = [ x = INT-> x | x =  FLOAT -> x ] -> Jq_number (float_of_string n)
     | "["; es = SELF ; "]" -> Jq_array es 
     | "{";  kvs = SELF ;"}" -> Jq_object kvs 

     | k= SELF; ":" ; v = SELF -> Jq_colon (k, v)
     | a = SELF; "," ; b = SELF -> Jq_comma (a, b)
     | -> Jq_nil  (* camlp4 parser epsilon has a lower priority *)

     ]];
END ;;

module AQ = Syntax.AntiquotSyntax 
module Q = Syntax.Quotation 
let destruct_aq s = 
  let pos = String.index s ':' in 
  let len = String.length s in 
  let name = String.sub s 0 pos in
  let code = String.sub s (pos+1) (len-pos-1) in 
  name, code

(** alternative*)  
let destruct_aq2  = function (RE (_* Lazy as  name ) ":" (_* as content)) -> name,content;;
\end{bluecode}

\begin{alternate}
let /(_* Lazy as x) ":" (_* as rest ) / = "ghsoghos:ghsogh: ghsohgo";;
val rest : string = "ghsogh: ghsohgo"

val x : string = "ghsoghos"  
\end{alternate}

\begin{bluecode}
let try /(_* Lazy as x) ":" (_* as rest ) / = "ghsoghosghsog ghsohgo"
in (x,rest)
with Match_failure _ -> ("","");;  
\end{bluecode}
notice that  Syntax.AntiquotSyntax.(parse\_expr,parse\_patt)
Syntax.(parse\_implem, parse\_interf)

\begin{bluecode}
        val parse_expr : Ast.loc -> string -> Ast.expr
        val parse_patt : Ast.loc -> string -> Ast.patt
    val parse_implem :
    val parse_interf :
\end{bluecode}

\begin{bluecode}
let aq_expander = object 
  inherit Ast.map as super 
  method expr = function 
    |Ast.ExAnt(_loc, s) -> 
      let n, c = destruct_aq s in 
      (** first round*)
      let e = AQ.parse_expr _loc c in 
      begin match n with 
        |"bool" -> <:expr< Jq_ast.Jq_bool $e$ >> (* interesting *)
        |"int" -> <:expr< Jq_ast.Jq_number (float $e$ ) >>
        |"flo" -> <:expr< Jq_ast.Jq_number $e$ >>
        |"str" -> <:expr< Jq_ast.Jq_string $e$ >>
        | "list" -> <:expr< Jq_ast.t_of_list $e$ >>
        |"alist" -> 
          <:expr<
            Jq_ast.t_of_list 
            (List.map (fun (k,v) -> Jq_ast.Jq_colon (Jq_ast.Jq_string k, v))
            $e$ )
          >>
        |_ -> e 
      end 
    |e -> super#expr e 
  method patt = function 
    | Ast.PaAnt(_loc,s) -> 
      let n,c = destruct_aq s in 
      AQ.parse_patt _loc c  (* ignore the tag *)
    | p -> super#patt p 
end 
module MetaExpr = struct 
  (** the generator scans all the types defined in the current module
      then generate code for the last-appearing recursive bundle
  *)
  let meta_float' _loc f = <:expr< $`flo:f$ >>
  include Camlp4Filters.MetaGeneratorExpr(Jq_ast)
end 
module MetaPatt = struct 
  let meta_float' _loc f = <:patt< $`flo:f$ >>
  include Camlp4Filters.MetaGeneratorPatt(Jq_ast)  
end 
let (|>) x f = f x 
let parse_quot_string _loc s = 
  let q = !Camlp4_config.antiquotations in 
  (** checked by the lexer to allow antiquotation 
      the flag is initially set to false, so antiquotations 
      appearing outside a quotation won't be parsed 
      *)
Camlp4_config.antiquotations := true ; 
let res =  MGram.parse_string  json_eoi _loc s in 
 Camlp4_config.antiquotations := q ; 
 res 
let expand_expr _loc _ s = 
  s 
  |> parse_quot_string _loc 
  |> MetaExpr.meta_t _loc 
  |> aq_expander#expr 
(* so it can appear in the toplevel *)
let expand_str_item _loc _ s = 
  (**insert an expression as str_item *)
   <:str_item@_loc< $exp: expand_expr _loc None s $ >>
let expand_patt _loc _ s  = 
  s 
  |> parse_quot_string _loc 
  |> MetaPatt.meta_t _loc 
  |> aq_expander#patt 
let _  = 
  Q.add "json" Q.DynAst.expr_tag expand_expr ;
  Q.add "json" Q.DynAst.patt_tag expand_patt ;
  Q.add "json" Q.DynAst.str_item_tag expand_str_item ;
  Q.default := "json"

\end{bluecode}
\begin{alternate}
MGram.parse_string json_eoi Loc.ghost "[1,2]";; 
 - : t = Jq_array (Jq_comma (Jq_number 1., Jq_number 2.)) 
MGram.parse_string json_eoi Loc.ghost "[1,2,]";;
- : t = Jq_array (Jq_comma (Jq_comma (Jq_number 1., Jq_number 2.), Jq_nil))
MGram.parse_string json_eoi Loc.ghost "1,2";;
- : t = Jq_comma (Jq_number 1., Jq_number 2.)
let alist = ["haha", <<1>>;"bob",<<3>>]  in <:json< [1 , $alist:alist$ ]>>;;
\end{alternate}

\begin{bluecode}
- : Json_anti.Jq_ast.t =
Json_anti.Jq_ast.Jq_array
 (Json_anti.Jq_ast.Jq_comma (Json_anti.Jq_ast.Jq_number 1.,
   Json_anti.Jq_ast.Jq_comma
    (Json_anti.Jq_ast.Jq_colon (Json_anti.Jq_ast.Jq_string "haha",
      Json_anti.Jq_ast.Jq_number 1.),
    Json_anti.Jq_ast.Jq_comma
     (Json_anti.Jq_ast.Jq_colon (Json_anti.Jq_ast.Jq_string "bob",
       Json_anti.Jq_ast.Jq_number 3.),
     Json_anti.Jq_ast.Jq_nil))))
   \end{bluecode}

\begin{alternate}   
let b = << $ << 1 >>  $ >>  = << 1 >>;;
val b : bool = true
\end{alternate}

\begin{bluetext}
<< $ << 1 >> $>> --> parsing (my parser)
Jq_Ant(_loc, "<< 1 >> ") --> lifting  (mechnical)
Ex_Ant(_loc, "<< 1 >>") --> parsing  (the host parser )
<:expr< Jq_number 1. >>   --> antiquot_expand (my anti_expander )
<:expr < Jq_number 1. >> 
*)
"json_anti.ml" : pp(camlp4of -filter meta)
<json_anti.{cmo,byte,native}> : pkg_dynlink, use_camlp4_full
  \end{bluetext}

\end{enumerate}
\item part 10 lexer \\
  Just follow the signature of module type Lexer is enough.
  generally you have to provide module
  Loc, Token, Filter, Error, and mk
  mk is essential

  \begin{bluecode}
val mk : unit -> Loc.t -> char Stream.t -> (Token.t * Loc.t ) Stream.t     
  \end{bluecode}

  the verbose part lies in that you have to use the Camlp4.Sig.Loc,
  usually you have to maintain a mutable context, so when you lex a
  token, you can query the context to get Loc.t. you can refer Jake's jq\_lexer.ml
  for more details. How about using lexer, parser all by myself?
  The work need to be done lies in you have to supply a plugin of type
  expand\_fun, which is \\
  \verb|type 'a expand_fun = Ast.loc -> string option -> string -> 'a|
  so if you dont use ocamllexer, why bother the grammar module, just
  use lex yacc will make life easier, and you code will run faster . 

\begin{bluecode}
type pos = {
  line : int;
  bol  : int;
  off  : int
};
type t = {
  file_name : string;
  start     : pos;
  stop      : pos;
  ghost     : bool
};
open Camlp4.PreCast 
module Loc = Camlp4.PreCast.Loc 
module Error : sig 
  type t 
  exception E of t 
  val to_string : t -> string 
  val print : Format.formatter -> t -> unit 
end =  struct
  type t = string 
  exception E of string 
  let print = Format.pp_print_string (* weird, need flush *)
  let to_string  x  =  x
end
let _ = 
  let module M = Camlp4.ErrorHandler.Register (Error) in ()
let (|> ) x  f = f x 
module Token : sig 
  module Loc : Camlp4.Sig.Loc 
  type t 
  val to_string : t -> string 
  val print : Format.formatter -> t -> unit 
  val match_keyword : string -> t -> bool 
  val extract_string : t -> string 
  module Filter : sig 
    (* here t refers to the Token.t *)
    type token_filter = (t,Loc.t) Camlp4.Sig.stream_filter 
    type t 
    val mk : (string->bool)-> t 
    val define_filter : t -> (token_filter -> token_filter) -> unit 
    val filter : t -> token_filter 
    val keyword_added : t -> string -> bool -> unit 
    val keyword_removed : t -> string -> unit 
  end
  module Error : Camlp4.Sig.Error  
end = struct 
  (** the token need not to be a variant with arms with KEYWORD
      EOI, etc, although conventional
  *)
  type t = 
    | KEYWORD of string 
    | NUMBER of string 
    | STRING of string 
    | ANTIQUOT of string * string 
    | EOI
  let to_string t = 
    let p = Printf.sprintf in 
    match t with 
      |KEYWORD s -> p "KEYWORD %S" s 
      |NUMBER s -> p "NUMBER %S" s 
      |STRING s -> p "STRING %S" s 
      |ANTIQUOT (n,s) -> p "ANTIQUOT %S: %S" n s 
      |EOI -> p "EOI"
  let print fmt x = x |> to_string |> Format.pp_print_string fmt 
  let match_keyword kwd = function 
    |KEYWORD k when kwd = k -> true 
    |_ -> false 

  let extract_string = function 
    |KEYWORD s | NUMBER s | STRING s -> s 
    |tok -> invalid_arg ("can not extract a string from this token : "
                         ^ to_string tok)

  module Loc = Camlp4.PreCast.Loc 
  module Error = Error 
  module Filter = struct 
    type token_filter = (t * Loc.t ) Stream.t -> (t * Loc.t) Stream.t 

    (** stub out *)    
    (** interesting *)
    type t = unit 

    (** the argument to mk is a function indicating whether 
        a string should be treated as a keyword, and the default 
        lexer uses it to filter the token stream to convert identifiers
        into keywords. if we want our parser to be extensible, we should
        take this into account 
    *)
    let mk _ = ()
    let filter _ x  = x
    let define_filter _ _ = ()
    let keyword_added _ _ _ = ()
    let keyword_removed _ _ = ()
  end 
end 
module L = Ulexing 
INCLUDE "/Users/bob/predefine_ulex.ml" 
(* let rec token  c = lexer  *)
(*   | eof -> EOI  *)
(*   | newline -> token *)
(** TOKEN ERROR LOC 
    mk : unit -> Loc.t -> char Stream.t -> (Token.t * Loc.t) Stream.t

    Loc.of_tuple : 
    string * int * int * int * int * int * int * bool -> 
    Loc.t
*)
    
  \end{bluecode}
\end{enumerate}

%%% Local Variables: 
%%% mode: latex
%%% TeX-master: "../master"
%%% End: 
