

\subsection{toplevel}

\begin{enumerate}

\item \verb|#directory ``_build'' ;; #directory ``+camlp4'' ;; #load ``...''|
\item trace
\item labels (ignore labels in function types)
\item \verb|warnings print_depth print_length|
\item hacking Toploop
  \begin{itemize}
  \item re-direct 


\begin{redcode}
Toploop.execute_phrase (bool->formatter->Parsetree.toplevel_phrase->bool)
Toploop.read_interactive_input
\end{redcode}
\begin{bluecode}
- : (string -> string -> int -> int * bool) ref = (* topdirs.cmi *)
\end{bluecode}



\begin{redcode}
  Hashtbl.keys Toploop.directive_table;;
\end{redcode}

\begin{bluecode}
print_depth use principal untrace_all load list trace show directory u cd install_printer print_length labels remove_printer camlp4o quit untrace thread camlp4r  
\end{bluecode}


\begin{redcode}
Topdirs.(dir_load,dir_use,dir_install_printer,dir_trace,dir_untrace,dir_untrace_all,load_file,dir_quit,dir_cd);;  
\end{redcode}

\begin{bluecode}
- : (Format.formatter -> string -> unit) *
    (Format.formatter -> string -> unit) *
    (Format.formatter -> Longident.t -> unit) *
    (Format.formatter -> Longident.t -> unit) *
    (Format.formatter -> Longident.t -> unit) *
    (Format.formatter -> unit -> unit) *
    (Format.formatter -> string -> bool) * (unit -> unit) * (string -> unit)
\end{bluecode}




\item store env

  \begin{bluecode}
let env = !Toploop.toplevel_env
... blabbla ...     
Toploop.toplevel_env := env     
\end{bluecode}
\begin{bluecode}
Toploop.initialize_toplevel_env ()  
\end{bluecode}
  \item \textbf{sample file  for references } in findlib


\begin{lstlisting}[caption=Toplevel Code Sample,label=toplevel]
(* For Ocaml-3.03 and up, so you can do: #use "topfind" and get a
 * working findlib toploop.
 * First test whether findlib_top is already loaded. If not, load it now.
 * The test works by executing the toplevel phrase "Topfind.reset" and
 * checking whether this causes an error.
 *)
let exec_test s =
  let l = Lexing.from_string s in
  let ph = !Toploop.parse_toplevel_phrase l in
  let fmt = Format.make_formatter (fun _ _ _ -> ()) (fun _ -> ()) in
  try
    Toploop.execute_phrase false fmt ph
  with
      _ -> false
in
if not(exec_test "Topfind.reset;;") then (
  Topdirs.dir_load Format.err_formatter "/Users/bob/SourceCode/ML/godi/lib/ocaml/pkg-lib/findlib/findlib.cma";
  Topdirs.dir_load Format.err_formatter "/Users/bob/SourceCode/ML/godi/lib/ocaml/pkg-lib/findlib/findlib_top.cma";
);;
\end{lstlisting}

    
  \item \href{file:/Users/bob/SourceCode/ML/godi/build/distfiles/findlib-1.2.7/src/findlib/topfind.ml}{topfind.ml} \\
    ideas : we can write \textbf{some utils} to check code later 
    yeah. A poor man's code search tool (in the library \verb|dir_top_level|)


\begin{alternate}
se;;
- : ?ignore_module:bool -> (string -> bool) -> string -> string list =
se ~ignore_module:false (FILTER _*  "char" space* "->" space* "bool") "String";;
\end{alternate}

\begin{lstlisting}
module Dont_use_this_name_ever :
    val contains : string -> char -> bool
    val contains_from : string -> int -> char -> bool
    val rcontains_from : string -> int -> char -> bool
    val filter : (char -> bool) -> string -> string
    module IString : sig type t = String.t val compare : t -> t -> int end
    module NumString : sig type t = String.t val compare : t -> t -> int end
    module Exceptionless :
    module Cap :
        val filter : (char -> bool) -> [> `Read ] t -> 'a t
        val contains : [> `Read ] t -> char -> bool
        val contains_from : [> `Read ] t -> int -> char -> bool
        val rcontains_from : [> `Read ] t -> int -> char -> bool
        module Exceptionless :
\end{lstlisting}


    

\begin{redcode}
Hashtbl.add
    Toploop.directive_table
    "require"
    (Toploop.Directive_string
       (fun s ->
	  protect load_deeply (Fl_split.in_words s)
       ))
;;
Hashtbl.add Toploop.directive_table "pwd"
(Toploop.Directive_none (fun _ -> 
  print_endline (Sys.getcwd ())));;
#pwd;;
\end{redcode}

\begin{bluecode}
/Users/bob/SourceCode/Notes
\end{bluecode}



  \end{itemize}
\end{enumerate}


%%% Local Variables: 
%%% mode: latex
%%% TeX-master: "master"
%%% End: 
