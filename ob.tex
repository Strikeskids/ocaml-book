
\section{ocamlbuild}

The reason for ocamlbuild in OCaml is to solve the complex scheme to
build camlp4. But it's very useful in other aspects

Your code is in the \verb|_build| directory. \verb|ocamlbuild| copies
the \verb|needed| source files and compiles them.

In \verb|_build|, \verb|_log| file logs detailed building process.

\verb|ocamlbuild| automatically creates a symbol link to the executable it
  in the current directory

hygiene rules at start up (.cmo, .cmi, or .o should appear
outside of the \verb|_build|) (\verb|-no-hygiene|)


Important Compile Falgs 

\begin{spacing}{0.5}\small
\begin{tabular}{|c|c|}
\hline
option & comment \\
\hline
-quiet & \\
-verbose <level> & \\
-documentation & show rules and flags for a specific \verb|_tags| file
\\
-clean & \\
-r & Traverse directories by default(true:traverse) \\
-I <path> & \\
-Is <path,...> & \\
-X <path> & ignore directory \\
-Xs <path,...> & \\
-lib <flag> & link to ocaml library \\
-libs <flag,...> & \\
-mod <module> & link to ocaml module \\
-mods & \\
-pkg <package> & link to \verb|ocaml findlib| package \\
-pkgs <...> & \\
-lflag  <flag> & ocamlc link flags \\
-lflags & \\
-cflag & ocamlc comple flags \\
-cflags & \\
-yaccflag & \\
-yaccflags & \\
-lexflag & \\
-lexflags & \\
-pp & preprocessing flagss\\
-tag <tag> & add to default tags \\
-tags & \\
-show-tags & \verb|ocamlbuild -show-tags target| \\
-ignore <module,...> & \\
-no-hygiene & \\
-no-plugin & \\
-just-plugin & just build myocamlbuild.ml \\
-use-menhir & \\
-use-jocaml & \\
-use-ocamlfild & \\
-build-dir & set build directory (implies no-links)\\
-install-lib-dir <path> & \\
-install-bin-dir & \\
-ocamlc <command> & set the ocamlc command \\
-ocamlopt  & \\
-ocamldoc & \\
-ocamlyacc & \\
-menhir & set the menhir tool (use it after -use-menhir)\\
-ocamllex & \\
-ocamlmktop & \\
-ocamlrun & \\
--  & supply arguments \\
\hline
  \end{tabular}
\end{spacing}


Simple Examples

\begin{enumerate}
\item \verb|ocamlbuild -quiet xx.native -- args|
\item \verb|ocamlbuild -quite -use-ocamlfind xx.native -- args|

\item   pass flags to \verb|ocamlc|  at compile timei.e.   
  \verb|-cflags -I,+lablgtk,-rectypes|

\item linking with \textbf{ external} libraries. i.e.
  \verb|-libs unix,num|.  You may need add the options below to make
  it work if this not in OCaml's default search path
  \verb|-cflags -I,/usr/local/lib/ocaml|
  \verb|-lflags -I,/usr/local/lib/ocaml|


\item \verb|mllib| file

  \begin{bluetext}
cat top_level.mllib    
Dir_top_level_util
Dir_top_level  
\end{bluetext}

then you can \verb|ocamlbuild top\_level.cma|, then you can use
\textbf{ocamlobjinfo} to see exactly which modules are compacted into
it.

\begin{bluetext}
ocamlobjinfo _build/top_level.cma | grep Unit  
Unit name: Dir_top_level_util
Unit name: Dir_top_level
\end{bluetext}

\item \verb|mlpack| file 
  hierachical packing 
\todo{mlpack}
\end{enumerate}

\verb|_tags| File

Every source has a set of tags .
\begin{bluetext}
bash-3.2$ocamlbuild -show-tags test.ml
Tags for "test.ml":
  {. extension:ml, file:test.ml, ocaml, pkg_camlp4.macro, pkg_menhirLib,
     pkg_ulex, predefine_ulex.ml, quiet, syntax_camlp4o, traverse, use_menhir .}
bash-3.2$ ocamlbuild -show-tags test.byte
Tags for "test.byte":
  {. byte, extension:byte, file:test.byte, ocaml, pkg_menhirLib, pkg_ulex,
     program, quiet, traverse, use_menhir .}
bash-3.2$ ocamlbuild -show-tags test.native
Tags for "test.native":
  {. extension:native, file:test.native, native, ocaml, pkg_menhirLib,
     pkg_ulex, program, quiet, traverse, use_menhir .}
\end{bluetext}
%$

By preceding a tag with a minus sign, one may \verb|remove| tags from one or
more files.

The built-in \verb|_tags| file
\begin{bluetext}
<**/*.ml>   or <**/*.mli> or <**/*.ml.depends> : ocaml 
<**/*.byte> : ocaml, byte, program 
<**/*.native>: ocaml, native, program
<**/*.cma>:ocaml, byte,library
<**/*.cmxa>:ocaml,native,library
<**/*.cmo>:ocaml,byte
<**/*.cmx>:ocaml,native
\end{bluetext}
<**/*.ml> means that .ml files in \emph{current dir or sub dir}. A
special tag made from the path name of the file relative to the
toplevel of the project is automatically defined for each file.  Just
as above \verb|test.ml| will be tagged \verb|file:test.ml|and also
\verb|extension:ml|

\begin{enumerate}
\item Including subdirectories \verb|include|
  \verb|<foo> or <bar> or <baz> : include|
  suppose you are calling 
  \verb|ocamlbuild foo/main.byte|, then use Foo, Foo.Bar and Foo.Baz
  in your code
\item grouping targets \verb|foo.itarget, foo.otarget|
  \begin{bluetext}
cat foo.itarget
main.native
main.byte 
stuff.docdir/index.html    
  \end{bluetext}

\item  packing \verb|foo.mlpack|
  \begin{bluetext}
cat foo.mlpack    
Bar
Baz
  \end{bluetext}
\item grouping \verb|foo.mllib|
\item preprocessing either \verb|-pp| or tags \verb|pp(cmd ...)|
\item debuggin and profiling either \verb|.d.byte, .p.native| or 
  \verb|true:debug|
\item documentation target
  
  create a file called \verb|foo.odocl|, then write the modules you
  want to document, then build the target \verb|foo.docdir/index.html|
  when you use -keep-code flag in myocamlbuild.ml, \textit{only}
  document of exposed modules are kept, not very useful
  \verb| flag ["ocaml"; "doc"] & S[A"-keep-code"];|
  ocamldep seems to  be \textbf{ lightweight}
 \verb|ocamlbuild -ocamldoc 'ocamlfind ocamldoc -keep-code' foo.docdir/index.html|
 It's weird when you have \verb|mli| file, \verb|-keep-code| does not
 work
\item glob patterns
\todo{glob patterns}
\end{enumerate}



\textbf{With lex yacc, ocamlfind }

\begin{enumerate}
\item \verb|.mll .mly| supported by default, \verb|{menhir (-use-menhir)| or add a line
  \verb|true : use_menhir|

\item add a line in tags file
  \verb|<*.ml> : pkg_sexplib.syntax, pkg_batteries.syntax, syntax_camlp4o|
  here \verb|syntax_camlp4o| is translated by myocamlbuild.ml to
 \verb|-syntax camlp4o| to pass to ocamlfind
 pkg needs \textbf{ocamlbuild plugin} support.
\end{enumerate}

Examples with Syntax extension 
\begin{bluetext}
<*.ml>: package(lwt.unix), package(lwt.syntax), syntax(camlp4o) -- only needs lwt.syntax when preprocessing
"prog.byte": package(lwt.unix)
\end{bluetext}

\begin{Verbatim}
<pa_*r.{ml,cmo,byte}> : pkg_dynlink , pp(camlp4rf ), use_camlp4_full

<*_ulex.{byte,native}> : pkg_ulex 

<*_ulex.ml> : syntax_camlp4o,pkg_ulex,pkg_camlp4.macro  

<*_r.ml>:syntax_camlp4r,pkg_camlp4.quotations.r,pkg_camlp4.macro,pkg_camlp4.extend 

pa_vector_r.ml:syntax_camlp4r,pkg_camlp4.quotations.r,pkg_camlp4.extend,pkg_sexplib.syntax

<pa_vector_r.{cmo,byte,native}>:pkg_dynlink,use_camlp4_full,pkg_sexplib 

<*_o.ml> : syntax_camlp4o,pkg_sexplib.syntax 

"map_filter_r.ml" : pp(camlp4r -filter map)

"wiki_r.ml" or "wiki2_r.ml"  : pp(camlp4rf -filter meta), use_camlp4_full


"wiki2_r.mli" : use_camlp4_full 
\end{Verbatim}

The .mli file also needs tags. For syntax extension, \textbf{order
  matters} , For more information, check out \textbf{
  camlp4/examples}. when you use pp flag, you need to specify the path
to \verb|pa_xx.cmo|, so symbol link may help.  Since 3.12,, you can
use \verb|-use-ocamlfind| to activate. ocamlfind predicates can be
activated with the \verb|predicate(...)| tag.
\begin{bluetext}
  <*.ml>: package(lwt.unix), package(lwt.syntax), syntax(camlp4o)
"prog.byte": package(lwt.unix)
\end{bluetext}


Interaction with git
\begin{bluetext}
_log
_build
*.native
*.byte
*.d.native
*.p.byte  
\end{bluetext}
ocamlbuild cares white space, \textbf{ take care when write tags file}


Rules 

A rule is composed of triple (Tags, Targets => Dependencies).

Principal 

\verb|ocamlbuild| looks for all rules that are valid for this target.
You can set \verb|-verbose 10| to get the backtrace in case of a
failure.

Plugins
\href{http://nicolaspouillard.fr/ocamlbuild/html/Signatures.PLUGIN.html}{Plugin
API}

There are 3 stages,(hygiene, options(parsing the command line
options), rules(adding the default rules to the system)). You can add
hooks to what you want.

\verb\{Before|After}_{options|hygiene|rules}\
To change the options, simply refer to the \verb|Options| module.

\begin{redcode}
sub_modules "Ocamlbuild_plugin";;
module This_module_name_should_not_be_used :
    module Pathname :
        module Operators :
    module Tags :
        module Operators :
    module Command :
    module Outcome :
    module String :
    module List :
    module StringSet :
    module Options :
    module Arch :
    module Findlib :  
\end{redcode}

Useful API,

\verb|Pathname.t,Tags.eltstring|
List the tags of a file \verb|tags_of_pathname|
Tag a file \verb|tag_file|
Untag a file \verb|tag_file "x.ml" ["-use_unix"]|
\verb|Arch.print_info|

\begin{redcode}
rule;;
- : string ->
    ?tags:string list ->
    ?prods:string list ->
    ?deps:string list ->
    ?prod:string ->
    ?dep:string ->
    ?stamp:string ->
    ?insert:[ `after of string | `before of string | `bottom | `top ] ->
    Ocamlbuild_plugin.action -> unit
= <fun>
\end{redcode}
The first arg is the name of the rule(unique required), \verb|~dep| is
the dependency, \verb|~prod| is the production. For example with
\verb|~dep:"%.ml" ~prod:"%.byte"|, you can produce ``bla.byte'' from
``bal.ml''. There are some predefined commands such as Unix
commands(cp,mv,...).

flag,dep
\begin{redcode}
flag ["ocaml"; "compile"; "thread"'] (A "-thread")  
\end{redcode}
It says tags \verb|ocaml, compile, thread| should become
\verb|-thread|

\begin{redcode}
type t =
  |Seq of t list
  (* A sequence of commands (like the `;' in shell) *)
  |Cmd of spec
  (* A command is made of command specifications (spec) *)
  |Echo of string list * pathname
  (* Write the given strings (w/ any formatting) to the given file *)
  |Nop
  (*The type t provides some basic combinators and command
    primitives. Other commands can be made of command specifications
    (spec).*)
type spec =
  |N (*No operation.	*)
  |S of spec list(* A sequence. This gets flattened in the last stages*)
  |A of string	(* An atom.*)
  |P of pathname(* A pathname.*)
  |Px of pathname
  (* A pathname, that will also be given to the call_with_target
     hook. *)
  |Sh of string
  (* A bit of raw shell code, that will not be escaped. *)
  |T of tags
  (* A set of tags, that describe properties and some semantics
     information about the command, afterward these tags will be replaced
     by command specs (flags for instance). *)
  |V of string
  (* A virtual command, that will be resolved at execution using
     resolve_virtuals *)
  |Quote of spec
  (* A string that should be quoted like a filename but isn't really
   one. *)
\end{redcode}

\mint{ocaml}|module Options| contains refs to be configured
\inputminted{ocaml}{code/ocamlbuild/options.mli}

Some Examples

\inputminted{ocaml}{code/ocamlbuild/myocamlbuild.2.ml}
\inputminted{ocaml}{code/ocamlbuild/myocamlbuild.3.ml}

\begin{bluetext}
 "bar.ml": camlp4o, use_openin
 <foo/*.ml> or <baz/**/*.ml>: camlp4r, use_openin
 "pa_openin.ml": use_camlp4, camlp4o  
\end{bluetext}

\inputminted{ocaml}{code/ocamlbuild/myocamlbuild.4.ml}
\inputminted{ocaml}{code/ocamlbuild/myocamlbuild.5.ml}

%%% Local Variables: 
%%% mode: latex
%%% TeX-master: "master"
%%% End: 
