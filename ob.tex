
\subsection{ocamlbuild}
\begin{enumerate}
\item directory hierarchy \\
code : \textit{\_build}
\begin{enumerate}
\item ob \emph{automatically creates a symbol link} to the executable it
  produces in the current directory
\item ob copies the sources and compiles them in \_build (default)
\item hygiene rules at start up (.cmo, .cmi, or .o should appear
  outside of the \_build) (-no-hygiene)
\item ob must be invoked in the root directory
\end{enumerate}

\item arguments

\begin{enumerate}
\item \textit{ocamlbuild -quite xx.native -- args}
\item \textit{ocamlbuild -quite -use-ocamlfind xx.native -- args}
\item -log -verbose -clean \\
  check \textit{\_build/\_log} file for detailed building process 
\item -cflags \\
  pass flags to \textbf{ocamlc} i.e. 
  -cflags -I,+lablgtk,-rectypes. (needed at compile time)
\item -lflags \\
  needed at linking time 
\item -libs \\
   linking with \textbf{ external} libraries. i.e. \emph{-libs unix,num}.
   you may need   \emph{-cflags -I,/usr/local/lib/ocaml}  \emph{-lflags -I,/usr/local/lib/ocaml} to make it work 
\item -use-ocamlfind
\item -pkgs oUnit

\item \textit{mllib} file

  \begin{redcode}
cat top_level.mllib    
\end{redcode}
\begin{bluecode}
Dir_top_level_util
Dir_top_level  
\end{bluecode}

then you can \textit{ocamlbuild top\_level.cma}, then you can use
\textit{ocamlobjinfo} to see exactly which modules are compacted into
it.

\begin{redcode}
ocamlobjinfo _build/top_level.cma | grep Unit  
\end{redcode}

\begin{bluecode}
Unit name: Dir_top_level_util
Unit name: Dir_top_level
\end{bluecode}


\end{enumerate}

\item with lex yacc, ocamlfind 

\begin{enumerate}
\item .mll .mly supported by default, \textit{menhir (-use-menhir)} or add a line
  \textit{true : use\_menhir}

\item add a line in tags file 
  \textit{<*.ml> : pkg\_sexplib.syntax, pkg\_batteries.syntax, syntax\_camlp4o} \\
  here \textit{syntax\_camlp4o} is translated by myocamlbuild.ml to \emph{-syntax camlp4o} to pass to ocamlfind

\item another typical tags file using \textbf{ syntax extension}
  

\begin{bluetext}
<*.ml>: package(lwt.unix), package(lwt.syntax), syntax(camlp4o) -- only needs lwt.syntax when preprocessing
"prog.byte": package(lwt.unix)
\end{bluetext}

  
\end{enumerate}

\item predicates
  
\begin{enumerate}
\item simple regexes \\
  <**/*.ml>   means that .ml files in \emph{current dir or sub dir}
  <**/*.ml> <**/*.mli> <**/*.mlpack> <**/*.ml.depends> : ocaml \\
  <**/*.byte> : ocaml, byte, program \\
  <*.ml> or <*.byte> or <*.native> : pkg\_oUnit \\
  <**/*.{native,byte}> : \verb|use_unix| \\
  <{batMutex,batRMutex}.{ml,mli}>: threads \\
  e1 or e2 , e1 and e2, not e, true ,false \\
  true:use\_menhir \\
\item  ocamlbuild cares white space, \textbf{ take care when write tags file}
\item foo.itarget
  

\begin{redcode}
bash$ cat foo.itarget
\end{redcode}

\begin{bluecode}
main.native
main.byte
stuff.docdir/index.html
\end{bluecode}

% $

\textbf{ ocamlbuild foo.otarget}

\item packing modules


\begin{redcode}
$ cat foo.mlpack
\end{redcode}

\begin{bluecode}
Bar
Baz 
\end{bluecode}

% $

\item document \\
  when you use -keep-code flag in myocamlbuild.ml, \textit{only} document of exposed modules are kept, not very useful \\
  \verb| flag ["ocaml"; "doc"] & S[A"-keep-code"];|
  ocamldep seems to be \textbf{ lightweight}

\item syntax extension \\ 
  Just for preprocessing, you can also use pp. \\
  \verb|<pa_*r.{ml,cmo,byte}> : pkg_dynlink , pp(camlp4rf ), use_camlp4_full| \\
  Here it not only use preporcessor, but also \textbf{ link} with it. \\
  Take ulex for example, for \textbf{ pre-processing} \\
  \verb|<*_ulex.ml> : syntax_camlp4o,pkg_ulex,pkg_camlp4.macro|,
  For \textbf{ linking} \\
  \verb|<*_ulex.{byte,native}> : pkg_ulex|. \\
  Normal for any revised syntax, you can say \\
  \verb|<*_r.ml>:syntax_camlp4r,pkg_camlp4.quotations.r,pkg_camlp4.macro,pkg_camlp4.extend| \\
  You can use \textbf{ several syntax extensions} together, as above. \\
  ``\verb|pa_vector_r.ml|'':\verb|syntax_camlp4r|,\verb|pkg_camlp4.quotations.r|,
  \verb|pkg_camlp4.extend|, \verb|pkg_sexplib.syntax|
  for \textbf{ preprocessing}, and \\
  \verb|<pa_vector_r.{cmo,byte,native}>:pkg_dynlink,use_camlp4_full,pkg_sexplib| for \textbf{ linking} . \\
  \textbf{ order matters}  \\
  For \textbf{ original} syntax, \verb|<*_o.ml> : syntax_camlp4o,pkg_sexplib.syntax| \\
  For \textbf{ filter} \verb|"map_filter_r.ml" : pp(camlp4r -filter map)|. and \\
  \verb|"wiki_r.ml" or "wiki2_r.ml"  : pp(camlp4rf -filter meta), use_camlp4_full|\\
  The .mli file also needs \verb|"wiki2_r.mli" : use_camlp4_full| \\
  for more information, check out \textbf{ camlp4/examples}. when you use pp flag, you need to specify the path to \verb|pa_xx.cmo|, so symbol link may help.
\end{enumerate}

\end{enumerate}
%%% Local Variables: 
%%% mode: latex
%%% TeX-master: "master"
%%% End: 
