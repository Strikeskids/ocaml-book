\section{module}

Module can be pased as a value

\begin{redcode}
module type ID = sig val id : 'a -> 'a end
let f m = 
    let module Id = (val m : ID) in 
    (Id.id 1, Id.id true);;

val f : (module ID) -> int * bool = <fun>  

f (module struct let id x = print_endline "ID!"; x end : ID);;

ID!
ID!
\end{redcode}
 
Here the argument \verb|m| is a module. This is already possible with
objects and records, but now modules are also allowed.
We introduce three syntaxes 
\mint{ocaml}|(module def : Sig) |
\mint{ocaml}|(val def : Sig)|
\mint{ocaml}|(module SIg)|
\begin{redcode}
module type DEVICE = sig end
let devices: (string, (module DEVICE)) Hashtbl.t = Hashtbl.create 18
module PDF = struct end
let _ = Hashtbl.add devices "PDF" (module PDF : DEVICE)
module Device =
  (val (Hashtbl.find devices "PDF") : DEVICE)

module type DEVICE = sig  end
val devices : (string, (module DEVICE)) Batteries.Hashtbl.t = <abstr>
module PDF : sig  end
module Device : DEVICE
\end{redcode}

Runtime choices, Type-safe plugins
\todo{Read the slides by Jacques Garrigue}

Parametric algorithms
\inputminted{ocaml}{lang/code/param.ml}
\begin{redcode}
  val average : (module Number with type t = 'a) -> 'a array -> 'a =
  <fun>

average (module struct type t = int let (+) = (+) let (/) = (/) let int = fun x -> x  end : Number with type t = int);;
- : int array -> int = <fun>
\end{redcode}
Notice \verb|with type t = int| is necessary here.

%%% Local Variables: 
%%% mode: latex
%%% TeX-master: "../master"
%%% End: 
