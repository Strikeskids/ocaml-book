\section{FFI}

\textbf{Basic Part} Only one thing to notice is that for bytecode, if
the number of arguments is greater than five, the C function's first
parameter will get an array containing all of the arguments, and the c
function's second parameter will get the number of arguments.

In this case, you may need to discriminate byte code from native
code. Syntax is like this:
\begin{ocamlcode}
external caml_name: type = "byte_code" "native_code"  
\end{ocamlcode}

The simple example is like this

\begin{listing}
  \inputminted[fontsize=\scriptsize,firstline=24,lastline=42]{c}{code/ffi/simple/plus_stubs.c}
  \caption{Plus Stubs}
  \label{lst:plus_stubs}
\end{listing}

\inputminted[fontsize=\scriptsize]{ocaml}{code/ffi/simple/plus.ml}

The building process is really easy combined with ocamlbuild

\begin{ocamlcode}
    dep ["link"; "ocaml"; "use_plus_stubs"] ["plus_stubs.o"];
    flag["link"; "ocaml"; "byte"] (S[A"-custom"]);
\end{ocamlcode}

The tags file is like this
\begin{bluetext}
<plus.{byte,native}> : use_plus_stubs  
\end{bluetext}
Notice that \verb|-custom| is only for byte link, native link will
link it by default. You can also enrich your \textit{runtime} without linking
to each byte file everytime.
\begin{bashcode}
  ocamlc -make-runtime -o new_ocamlrun progc.o a_c_library.a
  ocamlc -o vbcname.exe -use-runtime new_ocamlrun progocaml.cmo
\end{bashcode}
Instead of new runtime, you can also build a toplevlel linking c
functions
\begin{bashcode}
  ocamlmktop -custom -o ftop progc.o a_c_library.a ex.ml
\end{bashcode}

So you see, you have 4 choices, enrich enriovment, toplevel, bytecode,
native code

\subsection{Data representation}
\label{sec:data-representation}


\textbf{C Part} For the data discrimination, ocaml provides a lot of C
macros in \verb|<caml/values.h|.

It's illustrated very clearly in the header file:

\textit{word}: Four bytes on 32 and 16 bit architectures, eight bytes on 64 bit architectures.

\textit{long}: A C integer having \textit{the same number of bytes as}
a word.

\textit{val}: The ML representation of something. A long or a block or
a pointer outside the heap.  If it is a block, it is the (encoded)
address of an object, if it is a long, it is encoded as well.

\textit{block}: Something allocated.  It always has a header and some
fields or some number of bytes (a multiple of the word size).

\textit{field}: A word-sized val which is part of a block.

\textit{bp}: Pointer to the first \textit{byte} of a block.  (a char *)

\textit{op}: Pointer to the first \textit{field} of a block.  (a value
*)

\textit{hp}: a pointer to the header of a block.  (a char *)

\textit{int32}: Four bytes on all architectures.

\textit{int64}: Eight bytes on all architectures.

A block size is always a multiple of the word size, and at least one
word plus the header.

\textit{bosize}: Size (in bytes) of the "bytes" part.

\textit{wosize}: Size (in words) of the "fields" part.

\textit{bhsize}: Size (in bytes) of the block with its header.

\textit{whsize}: Size (in words) of the block with its header.

\textit{hd}: A header.

\textit{tag}: The value of the tag field of the header.

\textit{color}: The value of the color field of the header.  This is
for use only by the GC.

It's easy to discriminate whether is long or block by the lowest bit.
\inputminted[fontsize=\scriptsize,firstline=63,lastline=66]{/opt/godi/lib/ocaml/std-lib/caml/mlvalues.h}
That's why ocaml int is 31 bit.

%% don't edit it, it's organized in verbatim mode '
\begin{bluetext}
% 32 bit pointer
+----------------------------------------+---+---+
| pointer                                |0  | 0 |
+----------------------------------------+---+---+
% 64 bit pointer 
+------------------------------------+---+---+---+
| pointer                            | 0 | 0 | 0 |
+------------------------------------+---+---+---+

+--------------------------------------------+---+
| integer (31 or 63 bits)                    | 1 |
+--------------------------------------------+---+
\end{bluetext}
So if you need 32/64bit, you may need consult Bigarray.
There are other data conversion macroslike \verb|Type_val|,
\verb|Val_type|, which is straightforward.

For blocks the structure of the header is interesting:
For 16-bit and 32-bit architectures:
\begin{bluetext}
     +--------+-------+-----+
     |wosize  | color | tag |
     +--------+-------+-----+
   31         10  98  7     0
\end{bluetext}
For 64-bit architectures:
\begin{bluetext}
     +--------+-------+-----+
     |wosize  | color | tag |
     +--------+-------+-----+
    63        10 9 8  7     0
\end{bluetext}
Here are some more examples of lay out

\begin{bluetext}
+---------------+---------------+---------------+- - - - -
| header        | word[0]       | word[1]       | ....
+---------------+---------------+---------------+- - - - -
                                ^
                                |
                    pointer (a value)


+---------------+--------------------------------------+
| header        | 'a' 'b' 'c' 'd' 'e' 'f' '\O' '\1'    |
+---------------+--------------------------------------+
                ^
                |
           an OCaml string
+---------------+---------------+---------------+- - - - -
| header        | value[0]      | value[1]      | ....
+---------------+---------------+---------------+- - - - -
                ^
                |
          an OCaml array

+---------------+---------------+
| header        | arg[0]        |
+---------------+---------------+
                ^
                |
        a variant with one arg
+---------------+---------------+---------------+- - - - -
| header        | float[0]      | ....
+---------------+---------------+---------------+- - - - -
                ^
                |
       an OCaml float array
\end{bluetext}
So, it's easy to understand that 32 platform, it's wosize 22bits long
: the reason for the annoying 16MByte limit for string. The tag field is
interesting

The tag byte is multipurpose, in the variant-with-parameter example
above, it tells you which variant it is. In the string case, it
contains a little bit of runtime type information. In other cases it
can tell the gc that it's a lazy value or opaque data that the gc
should not scan
Table \ref{tab:run_time_rr} illustrate how tags are used to encode
different data types.
\begin{table}
  \centering
  \begin{tabular}{|p{3cm}|p{12cm}|}
    \hline 
    int, char & immediate value, shifted left by 1 bit, with LSB=1\\
    \hline 
    (), [], false & stored as OCaml int 0 (native 1) \\
    \hline 
    true & stored as OCaml int 1 \\
    \hline 
    type t = Foo | Bar | Baz
    (no parameters)  & stored as OCaml int 0,1,2 \\
    \hline 
    type t = Foo | Bar of int & the varient with no parameters are stored
    as OCaml int 0,1,2, etc. counting just the variants that have no parameters.
    The variants with parameters are stored as blocks, counting just the variants with
    parameters. The parameters are stored as words in the block itself. Note there is
    a limit around \textit{ 240 variants with parameters that applies to each type},
    but no limit on the number of variants without parameters you can have. \textit{this limit
      arises because of the size of the tag byte and the fact that some of high numbered tags are reserved} \\
    \hline 
    list [1;2;3] & This is represented as 1::2::3::[] where [] is a value in OCaml int 0,
    and h::t is a block with tag 0 and two parameters. This representation is exactly
    the same as if list was a variant \\
    \hline 
    tuples, struct and array & These are all represented identically, as a simple
    array of values, the tag is 0. The only difference is that an array can be allocated
    with variable size, but structs and tuples always have a fixed size.
    \\
    \hline
    struct or array where every elements is a float & These are treated as a special case.
    The tag has special value \verb|Dyn_array_tag| (254) so that the GC knows how to deal with
    these. \textit{ Note this exception does not apply to tuples that contains floats, beware
      anyone who would declare a vector as (1.0,2.0)}. \\
    \hline
    any string & strings are byte arrays in OCaml, but they have quite
    a clever representation to make it very efficient
    to get their length, and at the same time make them directly
    compatible with C strings. The tag is \verb|String_tag| (252).
    \\
    \hline 
  \end{tabular}
  \caption{Runtime type representation}
  \label{tab:run_time_rr}
\end{table}

So we can write a c function to walk through ocaml data as follows:
\inputminted[fontsize=\scriptsize,firstline=46,lastline=109]{c}{code/ffi/inspect/inspect_stubs.c}
You can see the last three ocaml values are exactly the same when in
the layer of run time.
\inputminted[fontsize=\scriptsize]{ocaml}{code/ffi/inspect/inspect.ml}
The dumped result is interesting, and worthwhile to digest.(output of bytecode)
\begin{bluetext}
immediate value
v as c integer: (7); as ocaml int: (3);as ocaml char: (^C)

memory block: size=1 --
float: 3 

immediate value
v as c integer: (195); as ocaml int: (97);as ocaml char: (a)

memory block: size=3 --
unknown structured block (tag=0):
....
....immediate value
....v as c integer: (3); as ocaml int: (1);as ocaml char: (^A)
....
....immediate value
....v as c integer: (3); as ocaml int: (1);as ocaml char: (^A)
....
....immediate value
....v as c integer: (1); as ocaml int: (0);as ocaml char: (^@)

memory block: size=3 --
unknown structured block (tag=0):
....
....immediate value
....v as c integer: (3); as ocaml int: (1);as ocaml char: (^A)
....
....immediate value
....v as c integer: (3); as ocaml int: (1);as ocaml char: (^A)
....
....immediate value
....v as c integer: (1); as ocaml int: (0);as ocaml char: (^@)

memory block: size=3 --
unknown structured block (tag=0):
....
....immediate value
....v as c integer: (3); as ocaml int: (1);as ocaml char: (^A)
....
....immediate value
....v as c integer: (3); as ocaml int: (1);as ocaml char: (^A)
....
....immediate value
....v as c integer: (1); as ocaml int: (0);as ocaml char: (^@)

memory block: size=3 --
unknown structured block (tag=0):
....
....immediate value
....v as c integer: (3); as ocaml int: (1);as ocaml char: (^A)
....
....immediate value
....v as c integer: (3); as ocaml int: (1);as ocaml char: (^A)
....
....immediate value
....v as c integer: (1); as ocaml int: (0);as ocaml char: (^@)

memory block: size=3 --
float array:
 1 1 0
memory block: size=2 --
unknown structured block (tag=0):
....
....memory block: size=1 --
....float: 1 
....
....memory block: size=2 --
....unknown structured block (tag=0):
........
........memory block: size=1 --
........float: 1 
........
........memory block: size=2 --
........unknown structured block (tag=0):
............
............memory block: size=1 --
............float: 0 
............
............immediate value
............v as c integer: (1); as ocaml int: (0);as ocaml char: (^@)

memory block: size=1 --
closure with 0 closure variables
....code pointer: 0x7ff451015578

memory block: size=1 --
closure with 0 closure variables
....code pointer: 0x7ff451015870

memory block: size=3 --
closure with 2 closure variables
....code pointer: 0x7ff45101586c
....
....memory block: size=1 --
....closure with 0 closure variables
........code pointer: 0x7ff451015870
....
....immediate value
....v as c integer: (5); as ocaml int: (2);as ocaml char: (^B)

memory block: size=4 --
closure with 3 closure variables
....code pointer: 0x7ff45101586c
....
....memory block: size=1 --
....closure with 0 closure variables
........code pointer: 0x7ff451015870
....
....immediate value
....v as c integer: (3); as ocaml int: (1);as ocaml char: (^A)
....
....immediate value
....v as c integer: (5); as ocaml int: (2);as ocaml char: (^B)

memory block: size=4 --
closure with 3 closure variables
....code pointer: 0x7ff451015844
....
....memory block: size=1 --
....closure with 0 closure variables
........code pointer: 0x7ff451015848
....
....immediate value
....v as c integer: (3); as ocaml int: (1);as ocaml char: (^A)
....
....immediate value
....v as c integer: (5); as ocaml int: (2);as ocaml char: (^B)

memory block: size=3 --
closure with 2 closure variables
....code pointer: 0x7ff451015680
....
....immediate value
....v as c integer: (3); as ocaml int: (1);as ocaml char: (^A)
....
....immediate value
....v as c integer: (5); as ocaml int: (2);as ocaml char: (^B)

memory block: size=1 --
closure with 0 closure variables
....code pointer: 0x7ff451015624

memory block: size=2 --
string: as c string: aghosh as ocaml string: aghosh(#0)aghoshgo(#0)

memory block: size=11 --
Not traced by GC tag 

memory block: size=2 --
Custom tag
\end{bluetext}
For partial evaluation under the \textit{bytecode}, it has some
special optimization, the field 0 is still code pointer, then if it
doesn't closure any variable, its block size 1, if it contains any
variable, its block size will be 2 + n. The second field is a code
pointer points to the real code. And both the \textit{first} field and
the \textit{second} field across the same name of partial function
could be \textit{shared}. (the first describes how to apply the
parameters, the second describes the real code). So a closure block
size could be $\{1,3,4,5, \cdots \}$.

Notice \verb|String_val| just do pointer arithmetic, does not keep the
semantics of ocaml string.(ocaml is null-terminated string(null
appended), but null is allowed to appear in the string) It was
illustrated as follows:
\inputminted[fontsize=\scriptsize,firstline=29,lastline=46]{c}{code/ffi/inspect/inspect_stubs.c}

\textbf{OCaml Part} We could also play with module Obj
\begin{ocamlcode}
Obj.("gshogh" |> repr |> tag);;
- : int = 252
let a = [|1;2;3|] in Obj.(a|>repr|>tag);;
- : int = 0
Obj.(a |> repr |> size);;
- : int = 3
\end{ocamlcode}

String has a clever algorithm
\begin{ocamlcode}
Obj.( "ghsoghoshgoshgoshgoshogh"|> repr |> size);;
- : int = 4 (4*8 = 32 )
"ghsoghoshgoshgoshgoshogh" |> String.length;;
24 (padding 8 bits)
\end{ocamlcode}

Like all heap blocks, strings contain a header defining
the size of the string in  machine words.

\begin{ocamlcode}
("aaaaaaaaaaaaaaaa"|>String.length);;
- : int = 16
# Obj.("aaaaaaaaaaaaaaaa"|>repr |> size);;
- : int = 3
\end{ocamlcode}
padding will tell you how many words are padded actually

The null-termination comes handy when passing a string to C, but is
not relied upon to compute the length (in Caml), allowing the string
to contain nulls.


\begin{ocamlcode}
repr : 'a -> t (id)
obj : t -> 'a (id)
magic : 'a -> 'b (id)
is_block : t -> bool = "caml_obj_is_block"
is_int : t -> bool = "%obj_is_int"
tag : t -> int ="caml_obj_tag" % get the tag field 
set_tag : t -> int -> unit = "caml_obj_set_tag"
size : t -> int = "%obj_size" % get the size field 
field : t -> int -> t = "%obj_field" % handle the array part 
set_field : t -> int -> t -> unit = "%obj_set_field"
double_field : t -> int -> float
set_double_field : t -> int -> float -> unit
new_block : int -> int -> t = "caml_obj_block"
dup : t -> t = "caml_obj_dup"
truncate : t -> int -> unit = "caml_obj_truncate"
add_offset : t -> Int32.t -> t = "caml_obj_add_offset"
marshal : t -> string 
Obj.(None |> repr |> is_int);;
- : bool = true
Obj.("ghsogho" |> repr |> is_block);;
- : bool = true
Obj.(let f x = x |> repr |> is_block in (f Bar, f (Baz 3)));;
- : bool * bool = (false, true)
\end{ocamlcode}


\subsection{Caveats}
\label{sec:ffi}
The input-output don't share their file buffers. try to use
\verb|fflush(stdout)|
  




