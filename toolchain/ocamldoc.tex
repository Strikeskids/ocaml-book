
\section{ocamldoc}
\label{sec:ocamldoc}



A special comment is associated to an element if it is placed before
or after the element.


A special comment before an element is associated to this element if :


There is no blank line or another special comment between the special
comment and the element. However, a regular comment can occur between
the special comment and the element.


The special comment is not already associated to the previous element.


The special comment is not the first one of a toplevel module.
A special comment after an element is associated to this element if
there is no blank line or comment between the special comment and the
element.

There are two exceptions: for type constructors and record fields in
type definitions, the associated comment can only be placed after the
constructor or field definition, without blank lines or other comments
between them. The special comment for a type constructor with another
type constructor following must be placed before the '|' character
separating the two constructors.



Some elements support only a subset of all @-tags. Tags that are not
relevant to the documented element are simply ignored. For instance,
all tags are ignored when documenting type constructors, record
fields, and class inheritance clauses. Similarly, a \verb|@param| tag on a
class instance variable is ignored

Markup language
\begin{bluetext}
text ::= (text_element)+
text_element ::=
| {[0-9]+ text}format text as a section header; the integer following
{ indicates the sectioning level.
  | {[0-9]+:label text} same, but also associate the name label to the
  current point. This point can be referenced by its fully-qualified
  label in a {! command, just like any other element.
    | {b text}set text in bold.
    | {i text}set text in italic.
    | {e text}emphasize text.
    | {C text}center text.
    | {L text}left align text.
    | {R text}right align text.
    | {ul list}build a list.
    | {ol list}build an enumerated list.
    | {{:string}text}put a link to the given address (given as a
    string) on the given text.
    | [string]set the given string in source code style.
    | {[string]}set the given string in preformatted source code
    style.
    | {v string v}set the given string in verbatim style.
    | take the given string as raw LATEX code.
      | {!string}insert a reference to the element named
      string. string must be a fully qualified element name, for
      example Foo.Bar.t. The kind of the referenced element can be
      forced (useful when various elements have the same qualified
      name) with the following syntax: {!kind: Foo.Bar.t} where kind
      can be module, modtype, class, classtype, val, type, exception,
      attribute, method or section.
      | {!modules: string string ...}insert an index table for the
      given module names. Used in HTML only.
      | {!indexlist}insert a table of links to the various indexes
      (types, values, modules, ...). Used in HTML only.
      | {^ text}set text in superscript.
      | {_ text}set text in subscript.
      | escaped_stringtypeset the given string as is; special
      characters ('{', '}', '[', ']' and '@') must beescaped by a '\'
      | blank_lineforce a new line.

list ::=
  | ({- text})+
  | ({li text})+
\end{bluetext}


Predefined tags

The folowing table gives the list of predefined @-tags, with their
syntax and meaning.
@author stringThe author of the element. One author by @author
tag. There may be several @author tags for the same element.


@deprecated textThe text should describe when the element was
deprecated, what to use as a replacement, and possibly the reason for
deprecation.


@param id textAssociate the given description (text) to the given
parameter name id. This tag is used for functions, methods, classes
and functors.


@raise Exc textExplain that the element may raise the exception Exc.


@return textDescribe the return value and its possible values. This
tag is used for functions and methods.


@see <url> textAdd a reference to the URL between '<' and '>' with the
given text as comment.


@see 'filename' textAdd a reference to the given file name (written
between single quotes), with the given text as comment.


@see ``document name'' textAdd a reference to the given document name
(written between double quotes), with the given text as comment.


@since stringIndicate when the element was introduced.


@before v textAssociate the given description (text) to the given
version v in order to document compatibility issues.


@version string: The version number for the element.
