


Directly supported in toplevel
Regular expression \emph{share} their own namespace.
\begin{enumerate}
\item compile
\begin{bluetext}
"test.ml" : pp(camlp4o -parser pa_mikmatch_pcre.cma)
<test.{cmo,byte,native}> : pkg_mikmatch_pcre
-- myocamlbuild.ml use default 
\end{bluetext}
\item toplevel
\begin{redcode}
ocaml
#camlp4o ;;
#require "mikmatch_pcre" ;; (* make sure to follow the order strictly *)
\end{redcode}
\item debug

  \begin{bluetext}
camlp4of -parser pa_mikmatch_pcre.cma -printer o test.ml
(* -no_comments does not work    *)
\end{bluetext}

\item structure \\
  regular expressions can be used to match strings, it must be preceded by
  the RE keyword, or placed between slashes (/../).

  \begin{bluecode}
    match ... with pattern -> ...
    function pattern -> ...
    try ... with pattern -> ... 
    let /regexp/ = expr in expr
    let try (rec) let-bindings in expr with pattern-match
    (only handles exception raised by let-bindings)
    MACRO-NAME regexp -> expr ((FILTER | SPLIT) regexp)
    
  \end{bluecode}

  \begin{alternate}
let x = (function (RE digit+) -> true | _ -> false) "13232";;
val x : bool = true
# let x = (function (RE digit+) -> true | _ -> false) "1323a2";;
val x : bool = true
# let x = (function (RE digit+) -> true | _ -> false) "x1323a2";;
val x : bool = false    
\end{alternate}

\begin{bluecode}
let get_option () = match Sys.argv with 
     [| _ |] -> None 
    |[| _ ; RE (lower+ as key) "=" (_* as data) |] -> Some(key,data)
    |_ -> failwith "Usage: myprog [key=val]";;
val get_option : unit -> (string * string) option = <fun>
\end{bluecode}

\begin{alternate}
let option = try get_option () with Failure (RE "usage"~) -> None ;;
val option : (string * string) option = None  
\end{alternate}


\item \textbf{sample regex}
  built in regexes
  \begin{bluetext}
    lower, upper, alpha(lower|upper), digit, alnum, punct
    graph(alnum|punct), blank,cntrl,xdigit,space
    int,float
    bol(beginning of line)
    eol
    any(except newline)
    bos, eos
  \end{bluetext}
  \begin{alternate}
let f = (function (RE int as x : int) -> x ) "132";;
val f : int = 132
let f = (function (RE float as x : float) -> x ) "132.012";;
val f : float = 132.012
let f = (function (RE lower as x ) -> x ) "a";;
val f : string = "a"
let src = RE_PCRE int ;;
val src : string * 'a list = ("[+\\-]?(?:0(?:[Xx][0-9A-Fa-f]+|(?:[Oo][0-7]+|[Bb][01]+))|[0-9]+)", [])
let x = (function (RE _* bol "haha") -> true | _ -> false) "x\nhaha";;
val x : bool = true
\end{alternate}

\begin{bluecode}
RE hello = "Hello!" 
RE octal  = ['0'-'7']
RE octal1 = ["01234567"]
RE octal2 = ['0' '1' '2' '3' '4' '5' '6' '7']
RE octal3 = ['0'-'4' '5'-'7']
RE octal4 = digit # ['8' '9']  (* digit is a predefined set of characters *)
RE octal5 = "0" | ['1'-'7']
RE octal6 = ['0'-'4'] | ['5'-'7']
RE not_octal = [ ^ '0'-'7'] (* this matches any character but an octal digit *)
RE not_octal' = [ ^ octal]  (* another way to write it *)
\end{bluecode}

\begin{redcode}
RE paren' = "(" _* Lazy ")"
(* _ is wild pattern, paren is built in *)
let p = function (RE (paren' as x )) -> x ;;
\end{redcode}

\begin{alternate}
p "(xx))";;
- : string = "(xx)"
# p "(x)x))";;
- : string = "(x)"
\end{alternate}

\begin{bluecode}
RE anything  = _*         (* any string, as long as possible *)
RE anything' = _* Lazy    (* any string, as short as possible *)
RE opt_hello  = "hello"?      (* matches hello if possible, or nothing *)
RE opt_hello' = "hello"? Lazy (* matches nothing if possible, or hello *)
RE num = digit+        (* a non-empty sequence of digits, as long as possible;
                          shortcut for: digit digit* *)
RE lazy_junk = _+ Lazy (* match one character then match any sequence
                          of characters and give up as early as possible *)

RE at_least_one_digit = digit{1+}     (* same as digit+ *)
RE at_least_three_digits = digit{3+}
RE three_digits = digit{3}
RE three_to_five_digits = digit{3-5}
RE lazy_three_to_five_digits = digit{3-5} Lazy

let test s = match s with 
    RE "hello" -> true 
  | _ -> false 
\end{bluecode}


It's important to know that matching process will try \textit{any} possible combination until
the pattern is matched. However the combinations are tried from left to right, and
repeats are either greedy or lazy. (greedy is default). laziness triggered by the presence
of the Lazy keyword.

\item fancy features of regex
  \begin{enumerate}[(a)]
  \item normal

    \begin{redcode}
let x = match "hello world" with
  RE "world" -> true
 | _ -> false;;
\end{redcode}

    \begin{bluecode}
val x : bool = false
  \end{bluecode}

\item pattern match syntax 
  (the let constructs can be used directly with a
  regexp pattern, but \textbf{let RE ... = ... }does not look nice, the
  sandwich notation (/.../) has been introduced )

  \begin{alternate}
Sys.ocaml_version;;
- : string = "3.12.1"
# RE num = digit + ;;
\end{alternate}

\begin{redcode}

RE num = digit + ;;
  
let  /(num as major : int ) "." (num as minor : int)

( "." (num as patchlevel := fun s -> Some (int_of_string s)) 
| ("" as patchlevel := fun s -> None ))

( "+" (_*  as additional_info := fun s -> Some s )
| ("" as additional_info  := fun s -> None )) eos

/ = Sys.ocaml_version ;;

\end{redcode}

we always use \textbf{as} to extract the information.      

\begin{bluecode}      
val additional_info : string option = None
val major : int = 3
val minor : int = 12
val patchlevel : int option = Some 1    
\end{bluecode}


\item File processing (Mikmatch.Text)

  \begin{bluecode}
    val iter_lines_of_channel : (string -> unit) -> in_channel -> unit
    val iter_lines_of_file : (string -> unit) -> string -> unit
    val lines_of_channel : in_channel -> string list
    val lines_of_file : string -> string list
    val channel_contents : in_channel -> string
    val file_contents : ?bin:bool -> string -> string
    val save : string -> string -> unit
    val save_lines : string -> string list -> unit
    exception Skip
    val map : ('a -> 'b) -> 'a list -> 'b list
    val rev_map : ('a -> 'b) -> 'a list -> 'b list
    val fold_left : ('a -> 'b -> 'a) -> 'a -> 'b list -> 'a
    val fold_right : ('a -> 'b -> 'b) -> 'a list -> 'b -> 'b
    val map_lines_of_channel : (string -> 'a) -> in_channel -> 'a list
    val map_lines_of_file : (string -> 'a) -> string -> 'a list
\end{bluecode}
\item \textbf{Mikmatch.Glob} (pretty useful)

  \begin{bluecode}
    val scan :
      ?absolute:bool ->
      ?path:bool ->
      ?root:string ->
      ?nofollow:bool -> (string -> unit) -> (string -> bool) list -> unit
    val lscan :
      ?rev:bool ->
      ?absolute:bool ->
      ?path:bool ->
      ?root:string list ->
      ?nofollow:bool ->
      (string list -> unit) -> (string -> bool) list -> unit
    val list :
      ?absolute:bool ->
      ?path:bool ->
      ?root:string ->
      ?nofollow:bool -> ?sort:bool -> (string -> bool) list -> string list
    val llist :
      ?rev:bool ->
      ?absolute:bool ->
      ?path:bool ->
      ?root:string list ->
      ?nofollow:bool ->
      ?sort:bool -> (string -> bool) list -> string list list
    \end{bluecode}

    here we want to get  \verb|~/.*/*.conf| file
    X.list (predicates corresponding to each layer .
    \begin{alternate}
let xs = let module X = Mikmatch.Glob in X.list ~root:"/Users/bob" [FILTER "." ; FILTER _* ".conf" eos ] ;;
val xs : string list = [".libfetion/libfetion.conf"]
\end{alternate}

\begin{redcode}
let xs =
  let module X = Mikmatch.Glob in
  X.list ~root:"/Users/bob" [const true; FILTER _* ".pdf" eos ]
  in print_int (List.length xs) ;;
\end{redcode}
\begin{bluecode}
455
\end{bluecode}


\item Lazy or Greedy 

  \begin{redcode}
match "acbde (result), blabla... " with 
RE _* "(" (_* as x) ")" -> print_endline x | _ -> print_endline "Failed";;
\end{redcode}
\begin{bluecode}
result
\end{bluecode}

\begin{redcode}
 match "acbde (result),(bla)bla... " with 
 RE _* Lazy "(" (_* as x) ")" -> print_endline x | _ -> print_endline "Failed";;
\end{redcode}
\begin{bluecode}
result),(bla
\end{bluecode}

\begin{alternate}
let / "a"? ("b" | "abc" ) as x / = "abc" ;; (* or patterns, the same as before*)
val x : string = "ab"
# let / "a"? Lazy ("b" | "abc" ) as x / = "abc" ;;
val x : string = "abc"
\end{alternate}

In place conversions of the substrings can be performed, using
either the predefined converters \textit{int, float}, or custom converters

\begin{alternate}
let z  = match "123/456" with RE (digit+ as x : int ) "/" (digit+ as y : int) -> x ,y ;;
val z : int * int = (123, 456)
\end{alternate}

Mixed pattern
\begin{alternate}
let z = match 123,45, "6789" with i,_, (RE digit+ as j : int) | j,i,_ -> i * j + 1;;
val z : int = 835048
\end{alternate}

\item Backreferences \\
  Previously matched substrings can be matched again using backreferences.

  \begin{alternate}
let z =  match "abcabc" with RE _* as x !x -> x ;;
val z : string = "abc"    
\end{alternate}

\item Possessiveness prevent backtracking 

  \begin{alternate}
let x = match "abc" with RE _* Possessive _ -> true | _ -> false;;
val x : bool = false    
  \end{alternate}

\item macros
  \begin{enumerate}

\item FILTER macro 
  \begin{alternate}
let f = FILTER int eos;;
val f : ?share:bool -> ?pos:int -> string -> bool = <fun>
# f "32";;
- : bool = true
# f "32a";;
- : bool = false    
\end{alternate}

\item REPLACE macro 
  \begin{alternate}
let remove_comments = REPLACE "#" _* Lazy eol -> "" ;;
val remove_comments : ?pos:int -> string -> string = <fun>
# remove_comments "Hello #comment \n world #another comment" ;;
- : string = "Hello \n world "
let x = (REPLACE "," -> ";;" ) "a,b,c";;
val x : string = "a;;b;;c"
\end{alternate}

\item REPLACE\_FIRST macro
\item SEARCH(\_FIRST) COLLECT COLLECTOBJ MACRO

  \begin{alternate}
let search_float = SEARCH_FIRST float as x : float -> x ;;
val search_float : ?share:bool -> ?pos:int -> string -> float = <fun>
search_float "bla bla -1.234e12 bla";;
- : float = -1.234e+12
let get_numbers = COLLECT float as x : float -> x ;;
val get_numbers : ?pos:int -> string -> float list = <fun>
get_numbers "1.2   83  nan  -inf 5e-10";;
- : float list = [1.2; 83.; nan; neg_infinity; 5e-10]
let read_file = Mikmatch.Text.map_lines_of_file (COLLECT float as x : float -> x );;
val read_file : string -> float list list = <fun>

(** Negative assertions *)
let get_only_numbers =  COLLECT < Not alnum . > (float as x : float) < . Not alnum > -> x

let list_words = COLLECT (upper | lower)+ as x -> x ;;
val list_words : ?pos:int -> string -> string list = <fun>
# list_words "gshogh sghos sgho ";;
- : string list = ["gshogh"; "sghos"; "sgho"]
RE pair = "(" space* (digit+ as x : int) space* ","  space* ( digit + as y : int ) space* ")";;
 # let get_objlist = COLLECTOBJ pair;;
val get_objlist : ?pos:int -> string -> < x : int; y : int > list =
  \end{alternate}  
\item SPLIT macro
  \begin{alternate}
let ys = (SPLIT space* [",;"] space* ) "a,b,c, d, zz;";;
val ys : string list = ["a"; "b"; "c"; "d"; "zz"]
let f = SPLIT space* [",;"] space* ;;
val f : ?full:bool -> ?pos:int -> string -> string list = <fun>
\end{alternate}

Full is false by default. When true, it considers the regexp
as a separator between substrings even if the first or the last one
is empty. will add some whitespace trailins
\begin{alternate}
f ~full:true "a,b,c,d;"  ;;
- : string list = ["a"; "b"; "c"; "d"; ""]
\end{alternate}
\item MAP macro (a weak lexer) (MAP regexp -> expr ) \\
  splits the given string into fragments: the fragments that do not match the pattern are returned as \textit{`Text s}. Fragments that match the pattern are replaced by the result of expr 

\begin{alternate}
let f = MAP ( "+" as x = `Plus ) -> x ;;
val f : ?pos:int -> ?full:bool -> string -> [> `Plus | `Text of string ] list =
let x =  (MAP ',' -> `Sep ) "a,b,c";;
val x : [> `Sep | `Text of string ] list = [`Text "a"; `Sep; `Text "b"; `Sep; `Text "c"]
\end{alternate}

\begin{redcode}
let f = MAP ( "+" as x = `Plus ) | ("-" as x = `Minus) | ("/" as x = `Div)
  | ("*" as x = `Mul) | (digit+ as x := fun s -> `Int (int_of_string s)) 
  | (alpha [alpha digit] +  as x := fun s -> `Ident s) -> x ;;
\end{redcode}

\begin{bluecode}
val f :
  ?pos:int ->
  ?full:bool ->
  string ->
  [> `Div
   | `Ident of string
   | `Int of int
   | `Minus
   | `Mul
   | `Plus
   | `Text of string ]
list = <fun>
\end{bluecode}
\begin{redcode}
# f "+-*/";;
\end{redcode}

\begin{bluecode}
- : [> `Div
     | `Ident of string
     | `Int of int
     | `Minus
     | `Mul
     | `Plus
     | `Text of string ]
    list
=
[`Text ""; `Plus; `Text ""; `Minus; `Text ""; `Mul; `Text ""; `Div; `Text ""]
\end{bluecode}

\begin{bluecode}
let xs = Mikmatch.Text.map (function `Text (RE space* eos) -> raise Mikmatch.Text.Skip | token -> token) (f "+-*/");;
val xs :
  [> `Div
   | `Ident of string
   | `Int of int
   | `Minus
   | `Mul
   | `Plus
   | `Text of string ]
  list = [`Plus; `Minus; `Mul; `Div]
\end{bluecode}


\item lexer (ulex is faster and more elegant)

  \begin{bluecode}
let get_tokens = f |- Mikmatch.Text.map (function `Text (RE space* eos)
-> raise Mikmatch.Text.Skip | `Text x -> invalid_arg x | x
-> x) ;;

val get_tokens :
  string ->
  [> `Div
   | `Ident of string
   | `Int of int
   | `Minus
   | `Mul
   | `Plus
   | `Text of string ]
  list = <fun>

get_tokens "a1+b3/45";;
- : [> `Div
     | `Ident of string
     | `Int of int
     | `Minus
     | `Mul
     | `Plus
     | `Text of string ]
    list
= [`Ident "a1"; `Plus; `Ident "b3"; `Div; `Int 45]  
\end{bluecode}

\item SEARCH macro (location)

  \begin{alternate}
let locate_arrows = SEARCH %pos1 "->" %pos2 ->  Printf.printf "(%i-%i)" pos1 (pos2-1);;
val locate_arrows : ?pos:int -> string -> unit = <fun>
# locate_arrows "gshogho->ghso";;
(7-8)- : unit = ()
let locate_tags = SEARCH "<" "/"? %tag_start (_* Lazy as tag_contents) %tag_end ">" -> Printf.printf "%s %i-%i" tag_contents tag_start (tag_end-1);;
\end{alternate}

\end{enumerate}

\item debug 
  \begin{alternate}
let src = RE_PCRE <Not alnum . > (float as x : float ) < . Not alnum > in print_endline (fst src);;
(?<![0-9A-Za-z])([+\-]?(?:(?:[0-9]+(?:\.[0-9]*)?|\.[0-9]+)(?:[Ee][+\-]?[0-9]+)?|(?:[Nn][Aa][Nn]|[Ii][Nn][Ff])))(?![0-9A-Za-z])
\end{alternate}




\item ignore the case
  \begin{alternate}
match "OCaml" with RE "O" "caml"~ -> print_endline "success";;
success    
\end{alternate}


\item zero-width assertions

  \begin{bluecode}
RE word =  < Not alpha . >    alpha+ < . Not alpha>
RE word' = < Not alpha . >    alpha+ < Not alpha >
\end{bluecode}

\begin{redcode}
RE triplet = <alpha{3} as x>
let print_triplets_of_letters = SEARCH triplet -> print_endline x
print_triplets_of_letters "helhgoshogho";;
\end{redcode}
\begin{bluecode}
hel
elh
lhg
hgo
gos
osh
sho
hog
ogh
gho
- : unit = ()
\end{bluecode}
\begin{redcode}
(SEARCH alpha{3} as x -> print_endline x ) "hello world";;
\end{redcode}

\begin{bluecode}
hel
wor
\end{bluecode}
\begin{redcode}
(SEARCH <alpha{3} as x> -> print_endline x ) "hello world";;
\end{redcode}
\begin{bluecode}
hel
ell
llo
wor
orl
rld
\end{bluecode}
\begin{redcode}
(SEARCH alpha{3} as x -> print_endline x ) ~pos:2 "hello world";;
\end{redcode}
\begin{bluecode}
llo
wor
\end{bluecode}





\item dynamic regexp 

  \begin{alternate}
let get_fild x = SEARCH_FIRST @x "=" (alnum* as y) -> y;;
val get_fild : string -> ?share:bool -> ?pos:int -> string -> string = <fun>
# get_fild "age" "age=29 ghos";;
- : string = "29"    
\end{alternate}



\item reuse \\ 
  using macro INCLUDE 

\item view patterns
  
  \begin{bluecode}
let view XY = fun obj -> try Some (obj#x, obj#y) with _ -> None ;;
val view_XY : < x : 'a; y : 'b; .. > -> ('a * 'b) option = <fun>
# let test_orign = function 
   %XY (0,0) :: _ -> true 
  |_ -> false 
;;
      val test_orign : < x : int; y : int; .. > list -> bool = <fun>    


let view Positive = fun x -> x > 0 
let view Negative = fun x -> x <= 0 

let test_positive_coords = function 
  %XY ( %Positive, %Positive ) -> true 
  | _ -> false

  (** lazy pattern is already supported in OCaml *)
let test x = match x with 
    lazy v -> v

type 'a lazy_list = Empty | Cons of ('a * 'a lazy_list lazy_t)
    

let f = fun (Cons (_ , lazy (Cons (_, lazy (Empty)) ) )) -> true ;;
let f = fun %Cons (x1, %Cons (x2 %Empty)) -> true  (* simpler *)
\end{bluecode}


implementation 
let view X = f is translated into:
   let view\_X = f

Similarly, we have local views:
let view X = f in ...

Given the nature of camlp4, this is the simplest solution that allows us to make views available to other modules, since they are just functions, with a standard name. When a view X is encountered in a pattern, it uses the view\_X function. The compiler will complain if doesn't have the right type, but not the preprocessor.

About inline views: since views are simple functions, we could insert functions directly in patterns. I believe it would make the pattern really difficult to read, especially since views are expected to be most useful in already complex patterns.


About completeness checking: our definition of views doesn't allow the compiler to warn against incomplete or redundants pattern-matching. We have the same situation with regexps. What we define here are incomplete or overlapping views, which have a broader spectrum of applications than views which are defined as sum types.

\item tiny use
  \begin{alternate}
  se (FILTER _* "map_lines_of_file" ) "Mikmatch";;
  val map_lines_of_file : (string -> 'a) -> string -> 'a list    
  \end{alternate}


\begin{bluecode}
let _  = Mikmatch.map_lines_of_file  
  (function x -> 
    match x with 
      | RE "\xbegin{bluecode}" -> "\n" ^ x 
      | RE "\xend{bluecode}" -> x ^ ``\n'' 
      | _  -> x  )
  "/Users/bob/SourceCode/Notes/ocaml-hacker.tex"  
  |> List.enum 
  |> File.write_lines "/Users/bob/SourceCode/Notes/ocaml-hacker-back-up.tex";;
\end{bluecode}

\end{enumerate}

\end{enumerate}

%%% Local Variables: 
%%% mode: latex
%%% TeX-master: "../master"
%%% End: 
