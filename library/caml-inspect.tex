

It's mainly used to debug programs or presentation.
\href{http://lambdamuesli.blogspot.com/}{blog}
\begin{enumerate}
\item usage
  \begin{bluetext}
#require "inspect";;
open Inspect ;;

Sexpr.(dump (test_data ()))
Sexpr.(dump dump) (** can dump any value, including closure *)
Dot.(dump_osx dump_osx)
\end{bluetext}



\item \textit{module Dot}
  \begin{bluetext}
    dump
    dump_to_file
    dump_with_formatter
    dump_osx
  \end{bluetext}
\item \textit{module Sexpr}
  \begin{bluetext}
    dump
    dump_to_file
    dump_with_formatter
  \end{bluetext}

\item principle \\
  OCaml values all share a \textit{common low-level} representation.
  The basic building block that is used by the runtime-system(which is
  written in the C programming languag) to represent any value in the
  OCaml universe is the value type. Values are always
  \textit{word-sized}. A word is either 32 or 64 bits
  wide(\textit{Sys.word\_size})

  A value can either be a pointer to a block of values in the OCaml
  heap, a pointer to an object outside of the heap, or an unboxed
  integer. Naturally, blocks in the heap are garbage-collected.

  To distinguish between unboxed integers and pointers, the system uses
  the least-significant bit of the value as a flag. If the LSB is set,
  the value is unboxed. If the LSB is cleared, the value is a pointer to
  some other region of memory. This encoding also explains why the int
  type in OCaml is only 31 bits wide (63 bits wide on 64 bit platforms).


  Since blocks in the heap are garbage-collected, they have strict
  structure constraints. Information like the tag of a block and its
  size(in words) is encoded in the header of each block.

  There are two categories of blocks with respect to the garbage collector:
  \begin{enumerate}
  \item Structured blocks \\
  May only contain well-formed values, as they are
  recursively traversed by the garbage collector.
  \item Raw blocks    \\
  are not scanned by the garbage collector, and can thus
  contain arbitrary values.
  \end{enumerate}
  Structured blocks have tag values lower than
  \textit{Obj.no\_scan\_tag}, while raw blocks have tags equal or
  greater than \textit{Obj.no\_scan\_tag}.
  

  The type of a block is its tag, which is stored in the block header.(\textit{Obj.tag})

  \begin{redcode}
Obj.(let f ()= repr |- tag in no_scan_tag, f () 0, f () [|1.;2.|], f
() (1,2) ,f ()[|1,2|]);;
\end{redcode}

\begin{bluecode}
- : int * int * int * int * int = (251, 1000, 254, 0, 0)    
\end{bluecode}

\begin{redcode}
se_str "_tag" "Obj";;  
\end{redcode}

\begin{bluecode}
    external tag : t -> int = "caml_obj_tag"
    external set_tag : t -> int -> unit = "caml_obj_set_tag"
    val lazy_tag : int
    val closure_tag : int
    val object_tag : int
    val infix_tag : int
    val forward_tag : int
    val no_scan_tag : int
    val abstract_tag : int
    val string_tag : int
    val double_tag : int
    val double_array_tag : int
    val custom_tag : int
    val final_tag : int
    val int_tag : int
    val out_of_heap_tag : int
    val unaligned_tag : int  
\end{bluecode}

\begin{enumerate}

\item \textit{0 to Obj.no\_scan\_tag-1} 
  A structured block (an array of Caml objects). Each field is a value.
\item \textit{Obj.closure\_tag}: A closure representing a functional value. The
first word is a pointer to a piece of code, the remaining words are
values containing the environment.
\item \textit{Obj.string\_tag}: A character string.
\item \textit{Obj.double\_tag}: A double-precision floating-point number.
\item \textit{Obj.double\_array\_tag}: An array or record of double-precision
floating-point numbers.
\item \textit{Obj.abstract\_tag}: A block representing an abstract datatype.
\item \textit{Obj.custom\_tag}: A block representing an abstract datatype with
  user-defined finalization, comparison, hashing, serialization and
  deserialization functions attached
\item \textit{Obj.object\_tag}: A structured block representing an object. The first
  field is a value that describes the class of the object. The second
  field is a unique object id (see \textit{Oo.id}). The rest of the block
  represents the variables of the object.
\item \textit{Obj.lazy\_tag, Obj.forward\_tag}: These two block types
  are used by the runtime-system to implement lazy-evaluation.
\item \textit{Obj.infix\_tag}: A special block contained within a
  closure block  
\end{enumerate}

\item representation

  For atomic types
  \begin{enumerate}
  \item int, char (ascii code) : Unboxed integer values
  \item float : Blocks with tag \textit{Obj.dobule\_tag}
  \item string : Blocks with tag \textit{Obj.string\_tag}
  \item int32, int64, nativeint : Blocks with \textit{Obj.custom\_tag}
  \end{enumerate}
  For Tuples and records: Blocks with tag 0
  \begin{alternate}
Obj.((1,2) |> repr |> tag);;
- : int = 0    
\end{alternate}
For normal array(except float array), Blocks with tag 0

For Arrays and records of floats: Block with tag
\textit{Obj.double\_array\_tag}

For concrete types, 
\begin{enumerate}
\item Constant ctor : Represented by unboxed integers(0,1,...).
\item Non-Constant ctor: Block with a tag lower than
  \textit{Obj.no\_scan\_tag} that encodes the constructor, numbered in
  order of declaration, starting at 0.
\end{enumerate}

For objects: Blocks with tag \textit{Obj.object\_tag}. The first field
refers to the class of the object and its associated method suite. The
second field contains a unique object ID. The remaining fields are the
instance variables of the object.

For polymorphic variants: Variants are similar to constructed
terms. There are a few differences
\begin{enumerate}
\item Variant constructors are identified by their hash value
\item Non-constant variant constructors are not flattened. They are
  always block of size 2, where the first field is the hash. The
  second field can either contain a single value or a pointer to
  another structured block(just like a tuple)
\end{enumerate}

\end{enumerate}


%%% Local Variables: 
%%% mode: latex
%%% TeX-master: "../master"
%%% End: 
